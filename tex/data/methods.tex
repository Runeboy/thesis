\section{Methods}
The process of verifying or falsifying the proposed hypotheses will compose of several steps, as accounted for here. 

\subsection{Scientific reference points}

First, a flyover of relevant and similar research literature will provide a reference point that may help better understand any challenges and approaches that could also be applicable to the particular problem at hand.

\subsection{Choice of technology}

Next, a specific mobile device and platform will be selected for developing a prototype. This includes acquiring the technical knowledge and practical programming skills required in order to create and deploy an application that implements the algorithm for \AirSwipe\ interactions. In order to implement the actual \AirSwipe\ algorithm, it will be necessary to get familiar with a commercial technology for tracking hand movements within a three-dimensional space and how this data can be relayed to the mobile device executing the algorithm.  

\subsection{Practical evaluation}

Lastly, the end result needs to be evaluated through an experimental setup that allows for testing whether or not the hypothesis holds true, preferably using a reasonable number of participating volunteers. 


\begin{comment}

\section{Expected analysis and results}

It is expected that the experiment will provide insight into whether or not interaction in off-screen space improves the user experience, given the particular approach taken. In addition, an analysis of how participants perceive and interact with the off-screen space through \AirSwipe\ gestures may lead to a subsequent exploration of a sub-hypothesis and possible refinement of the prototype. For instance, whether or not off-screen space, when implemented on a small portable device, correlates perfectly well with a simple imaginary extension of the display plane is an open question - one that is best explored through careful analysis of experimental data generated by actual human behavior. 

\end{comment}

