%% TODO CALC MEAN AND STD
% evan: 38
% rose: 40
% leah: 25
% lily: 20
% mike: 22

% (38+40+25+20+22) = mean 29, std=9.3274

 %pgfplotsinvokeforeach

%\tracinggroups=0

\documentclass[11.5pt]{article}

%\usepackage{color}
%\usepackage[dvipsnames]{xcolor}
\usepackage[usenames,dvipsnames,svgnames,table,rgb]{xcolor}

%\def\expParticipants{unnamed}
%\def\expParticipants{lily,leah,unnamed}

%\def\expParticipants{evan,rose}

%\def\expParticipants{lily,leah,evan,rose}


%\def\expParticipants{evan,rose,Lily,Lily,leah,evann,leah,rosee,soniaa}
%\def\expParticipants{evan,rose,Lily,Lily}
%\def\expParticipants{lily,leah}


\def\baselinePlotYMax{50}
%\def\baselinePlotYMax{}

%evan,rose,Lily,Lily,leah,evann,leah,rosee,soniaa,mikee,mikee

%\def\expParticipants{evan,rose,Lily,Lily,leah,evann,leah,rosee,soniaa,mikee,mikee}
\def\expParticipants{evan,rose,Lily,leah,evann,rosee,soniaa,mikee}


% FILE PATHS

\newcommand\datafilepath[1]{data/exp1/#1}

\def\expOneCsvFilepath{data/exp1/TrialSesssions-(\expParticipants).csv}
%\def\expTwoPolarCsvFilepath{data/exp1/TrialSesssions-(\expParticipants).csv}
\def\expTwoCsvFilepath{data/exp1/TrialSesssions-(E2-\expParticipants).csv}
\def\expTwoPolarCsvFilepath{data/exp1/TrialSesssions-(E2-mean,target,\expParticipants).csv}


%\def\expThreeCsvFilepath{data/exp1/TrialSesssions-(E2-\expParticipants).csv}
\def\expThreeCsvFilepath{data/exp1/TrialSesssions-(E3-mean,target,\expParticipants).csv}



\def\csvSep{tab}

\tolerance=300

\def\strangeScale{1/10}




\def\meanColor{gray!80}

\def\myGreen{green!80!black}
\def\myRed{red!100!black}
\def\myBlue{blue!90!black}

%\def\myCyan{GreenYellow!80!black}
%\def\myMagenta{BlueViolet!80!black}
%\def\myYellow{YellowOrange!90!black}
\def\myCyan{cyan}
\def\myMagenta{magenta}
\def\myYellow{yellow}
\def\myOrange{orange}

\def\myCyanMeanColor{\myCyan}
\def\myMagentaMeanColor{\myMagenta}
\def\myYellowMeanColor{\myYellow}


\def\themeColorLight{blue!13!white}
\def\themeColorDark{blue!50!white!80!black}



%
%\definecolor{lilyColor}{hsb}{0.0, 1, 1}
%\definecolor{leahColor}{hsb}{0.33, 1, 1}
%\definecolor{roseColor}{hsb}{0.66, 1, 1}
%\definecolor{evanColor}{hsb}{0.75, 1, 1}
%
%\def\lilyColor{lilyColor}
%\def\leahColor{leahColor}
%\def\roseColor{roseColor}
%\def\evanColor{evanColor}

\def\mikeColor{MidnightBlue}
\def\lilyColor{\myRed}
\def\leahColor{\myGreen}
\def\roseColor{violet}
\def\evanColor{\myBlue}
\def\soniaColor{\myOrange}
\def\unnamedColor{pink}
%\def\meanColor{\myRed}
\def\stdColor{brown}





\def\trialOvershootTravelRatioColor{Magenta}	
\def\trialInertialDistanceTraveledRatioColor{Yellow!50!Orange}	
\def\trialTouchCountColor{Cyan!80!blue}

%\def\expOneCsvFilepath{data/exp1/TrialSesssions.csv}




%\pgfplotstableread{}{\expTwoData}
\definecolor{target10}{hsb}{0.0, 0.8, 1}
\definecolor{target20}{hsb}{0.2, 0.8, 1}
\definecolor{target30}{hsb}{0.4, 0.8, 1}
\definecolor{target40}{hsb}{0.6, 0.8, 1}
\definecolor{target50}{hsb}{0.8, 0.8, 1}
\definecolor{target-10}{hsb}{0.0, 0.8, 1}
\definecolor{target-20}{hsb}{0.2, 0.8, 1}
\definecolor{target-30}{hsb}{0.4, 0.8, 1}
\definecolor{target-40}{hsb}{0.6, 0.8, 1}
\definecolor{target-50}{hsb}{0.8, 0.8, 1}

\newcommand\targetColor[1]{target#1}


\definecolor{angle0}{hsb}{0, 0.8, 1}
\definecolor{angle30}{hsb}{0.16, 0.8, 1}
\definecolor{angle60}{hsb}{0.33, 0.8, 1}
\definecolor{angle90}{hsb}{0.5, 0.8, 1}
\definecolor{angle120}{hsb}{0.66, 0.8, 1}
\definecolor{angle150}{hsb}{0.83, 0.8, 1}

\definecolor{angle180}{hsb}{0, 0.8, 1}
\definecolor{angle210}{hsb}{0.16, 0.8, 1}
\definecolor{angle240}{hsb}{0.33, 0.8, 1}
\definecolor{angle270}{hsb}{0.5, 0.8, 1}
\definecolor{angle300}{hsb}{0.66, 0.8, 1}
\definecolor{angle330}{hsb}{0.83, 0.8, 1}


\newcommand\angleColor[1]{angle#1}













\usepackage{ifthen}




%\pgfplotstableread{\expOneCsvFilepath}\expOneData


%\setcounter{secnumdepth}{3}

\def\dotNetDoubleLimit{1.7976931348623157 \times 10^{308}}






\def\true{\string TRUE}
\def\false{\string FALSE}

\def\isFinalBuild{\false} % set to true on final build

\newcommand\ifIsFinalBuildElse[2]{
	\ifthenelse{\equal{\isFinalBuild}{\true}}{#1}{#2}
}



%\def\ifBuild_#1{\\#1}

\def\IFBUILDSTART#1\IFBUILDEND{
	%	#1
}

\newcommand\drawWithBuildOption[2] {
	\draw[#1]
%	\draw[#1,#2]
}

\newcommand\runeshade[3]{
	\draw[#1, fill=#2]	
%		\draw[#1, top color=#1!#2]			
}


%
%\newcommand\projectionPointOptions{color=orange}
%\def\projectedPointOptions{color=gray}
%\def\projectionLineOptions{->, shorten <= 1pt, shorten >= 1pt, color = gray}
%
%


\def\colorTopInput{green!40}
\def\colorRightInput{blue!20}
\def\colorBottomInput{red!20}
\def\colorLeftInput{orange!50!yellow!40}


\def\colorTop{green}
\def\colorRight{blue!60}
\def\colorBottom{red!60}
\def\colorLeft{orange!70!yellow!50}


%\def\scatterClasses{{
%				lily={mark=*,\lilyColor,line width=0mm},
%				Lily={mark=*,\lilyColor,line width=0mm},
%				leah={mark=*,\leahColor,line width=0mm,xshift=-0.05cm},
%				evan={mark=*,\evanColor,line width=0mm,xshift=0.05cm},		
%				evann={mark=*,\evanColor,line width=0mm,xshift=0.05cm},		
%				rose={mark=*,\roseColor,line width=0mm,xshift=0.1cm},		
%				rosee={mark=*,\roseColor,line width=0mm,xshift=0.1cm},		
%				unnamed={mark=*,\unnamedColor,line width=0mm,xshift=0.15cm},		
%				mean={mark=none,\meanColor,line width=0mm},
%				%			mean={mark=none,\meanColor,line width=0mm},
%				std={mark=none,\stdColor,line width=0mm}
%			}}



%%%%%%%%%%%%%%%%%%%%%%%%%%%%%%%%%%%%%%%%%%%%%%%%%%%%%%
%%%% DEFS END %%%%%%%%%%%%%%%%%%%%%%%%%%%%%%%%%%%%%%%%%%
%%%%%%%%%%%%%%%%%%%%%%%%%%%%%%%%%%%%%%%%%%%%%%%%%%%%%%


% This is "sig-alternate.tex" V2.0 May 2012
% This file should be compiled with V2.5 of "sig-alternate.cls" May 2012
%
% This example file demonstrates the use of the 'sig-alternate.cls'
% V2.5 LaTeX2e document class file. 

%\documentclass{sig-alternate}

\usepackage{changepage}
\usepackage{excludeonly}

\usepackage{scalerel}

\usepackage{multicol}
\usepackage[utf8]{inputenc}	
\usepackage{color}
\usepackage{lipsum}
%\usepackage{xcolor}
\usepackage[colorlinks,linkcolor=navy,citecolor=blue]{hyperref}
\usepackage{verbatim}
\usepackage{textcomp}
\usepackage{tabularx}
\usepackage{csquotes}
%\usepackage{comment}
\usepackage{array}
\usepackage{geometry}
\usepackage{gensymb} % was


%\documentclass{sig-alternate}
%\usepackage{multicol}
\usepackage[utf8]{inputenc}
\usepackage{color}
%\usepackage{lipsum}
\usepackage[colorlinks,linkcolor=navy,citecolor=blue]{hyperref}
%\usepackage{verbatim}
%\usepackage{textcomp}
%\usepackage{tabularx}
%\usepackage{csquotes}
\usepackage{comment}
\usepackage{array}
\usepackage{geometry}
\usepackage{amsmath}
\usepackage[toc,page]{appendix}
%\usepackage[english]{babel}
\usepackage{graphicx}
\usepackage{mathtools}  
\usepackage{environ}  
\usepackage{ulem}
\usepackage{tikz}
%\usetikzlibrary{positioning,fadings,through}
%\usetikzlibrary{positioning,fadings}
\usetikzlibrary{calc,patterns,angles,quotes,shapes,positioning,shapes.misc,fadings}

\tikzset{cross/.style={cross out, draw=black, fill=none, minimum size=2*(#1-\pgflinewidth), inner sep=0pt, outer sep=0pt}, cross/.default={2pt}}


\usetikzlibrary{arrows}
\usepackage{pgfplots}
\usepackage{pgfplotstable,xstring}
\usepackage{pdfpages}
\usepackage{nicefrac}
\usepackage{twoopt}

\usepackage{datatool} % datatool,xfor, substr, fp


\usepackage{amsfonts}

\usepgfplotslibrary{fillbetween,patchplots,statistics,groupplots}
\usetikzlibrary{patterns,pgfplots.polar}
\usepackage{courier}

%\usepackage[document]{ragged2e} % imports the package ragged2e and left-justifies the text
%\raggedright 

\usepackage{bbm} % bbm, bbm-macros




\usepackage{csvsimple}




\usepackage{array}





\newcolumntype{L}[1]{>{\raggedright\let\newline\\\arraybackslash\hspace{0pt}}m{#1}}
\newcolumntype{C}[1]{>{\centering\let\newline\\\arraybackslash\hspace{0pt}}m{#1}}
\newcolumntype{R}[1]{>{\raggedleft\let\newline\\\arraybackslash\hspace{0pt}}m{#1}}

\newcommand{\fn}[1] {\footnote{#1}}

\newcommand{\AirSwipe}{Air-Swipe}

\definecolor{navy}{HTML}{000080} 

\geometry{
	letterpaper,
	left=20mm,
	right=20mm,
	top=23mm,
	bottom=27mm,
}


\setlength\columnsep{20pt}


\renewenvironment{abstract}
{\small
	\begin{center}
		\bfseries \abstractname\vspace{-.5em}\vspace{0pt}
	\end{center}
	\list{}{%
		\setlength{\leftmargin}{50mm}% <---------- CHANGE HERE
		\setlength{\rightmargin}{\leftmargin}%
	}%
	\item\relax}
{\endlist}


\makeatletter
\newcommand{\runedo}[2][ $\triangleright$ ]{%
	\def\nextitem{\def\nextitem{#1}}%
	\@for \el:=#2\do{\nextitem\el}%
}
\makeatother


%\pgfplotsset{
%	discard if/.style 2 args={
%		x filter/.code={
%			\edef\tempa{\thisrow{#1}}
%			\edef\tempb{#2}
%			\ifx\tempa\tempb
%			\def\pgfmathresult{inf}
%			\fi
%		}
%	},
%	discard if not/.style 2 args={
%		x filter/.code={
%			\edef\tempa{\thisrow{#1}}
%			\edef\tempb{#2}
%			\ifx\tempa\tempb
%			\else
%			\def\pgfmathresult{inf}
%			\fi
%		}
%	}
%}


\makeatletter
\pgfplotstableset{
	tablefilter in/.style 2 args={
		row predicate/.append code={
			\def\pgfplotstable@loc@TMPd{\pgfplotstablegetelem{##1}{#1}\of}
			\expandafter\pgfplotstable@loc@TMPd\pgfplotstablename
			\edef\tempa{\pgfplotsretval}
			\edef\tempb{#2}
			\ifx\tempa\tempb
			\else
			\pgfplotstableuserowfalse
			\fi
		}
	},
	tablefilter out/.style 2 args={
		row predicate/.append code={
			\def\pgfplotstable@loc@TMPd{\pgfplotstablegetelem{##1}{#1}\of}
			\expandafter\pgfplotstable@loc@TMPd\pgfplotstablename
			\edef\tempa{\pgfplotsretval}
			\edef\tempb{#2}
			\ifx\tempa\tempb
			\pgfplotstableuserowfalse
			\else
			\fi
		}
	}
}
\makeatother





\pgfplotsset{
	filter multiple/.style args={#1 equals #2 or #3}{
		x filter/.append code={
			\edef\tempa{\thisrow{#1}}
			\edef\tempb{#2}
			\edef\tempc{#3}
			\ifx\tempa\tempb
			\else
				\ifx\tempa\tempc
				\else
					\def\pgfmathresult{inf}
				\fi
			\fi
%			\edef\tempb{\thisrow{#1}}
%			\edef\tempa{#2}
%			\ifx\tempa\tempb
%				\edef\tempa{#3}
%%				\edef\tempb{#3}
%				\ifx\tempa\tempb
%				\else
%					\def\pgfmathresult{inf}
%				\fi
%			\else
%				\def\pgfmathresult{inf}
%			\fi
		}
	},
	interpret as polar/.style={
		x filter/.code=\pgfmathparse{cos(rawx)*rawy},
		y filter/.code=\pgfmathparse{sin(rawx)*rawy}
	},
	filter in/.style 2 args={
		x filter/.append code={
			\edef\tempa{\thisrow{#1}}
			\edef\tempb{#2}
			\ifx\tempa\tempb
			\else
			\def\pgfmathresult{inf}
			\fi
		}
	},
	filter out/.style 2 args={
		x filter/.append code={
			\edef\tempa{\thisrow{#1}}
			\edef\tempb{#2}
			\ifx\tempa\tempb
			\def\pgfmathresult{inf}
			\fi
		}
	},
	discard if/.style 2 args={
		x filter/.append code={
			\edef\tempa{\thisrow{#1}}
			\edef\tempb{#2}
			\ifx\tempa\tempb
			\def\pgfmathresult{inf}
			\fi
		}
	},
	discard if not/.style 2 args={
		x filter/.append code={
			\edef\tempa{\thisrow{#1}}
			\edef\tempb{#2}
			\ifx\tempa\tempb
			\else
				\def\pgfmathresult{inf}
			\fi
		}
	},
  discard ifnum/.style 2 args={
  	x filter/.append code={
  		\ifdim\thisrow{#1} pt=#2pt
  		\def\pgfmathresult{inf}
  		\fi
  	}
  },
  discard if notnum/.style 2 args={
  	x filter/.append code={
  		\ifdim\thisrow{#1} pt=#2pt
  		\else
  		\def\pgfmathresult{inf}
  		\fi
  	}
  },	
	reduce to/.style args={every#1except between values#2and#3}{%
		/pgfplots/x filter/.code={%
			\let\pgfmathreserved\pgfmathresult
			\def\myswitch{1}%
			\pgfmathparse{##1>#2}%
			\ifpgfmathfloatcomparison
			\pgfmathparse{##1<#3}%
			\ifpgfmathfloatcomparison
			\def\myswitch{0}%
			\fi%
			\fi%
			\let\pgfmathresult\pgfmathreserved
			\ifnum1=\myswitch%
			\pgfmathsetmacro\temp{int(mod(\coordindex,#1))}%
			\ifnum0<\temp
			\let\pgfmathresult\pgfutil@empty
			\fi%
			\fi%
		}
	} 
}

% -------------------------------------


\makeatletter


% Set some defaults 
\tikzset{
	plane max x/.initial=2,
	plane max y/.initial=2,
	plane max z/.initial=2
}

\tikzset{plane/.style={fill opacity=0.5}}

% Define a plane.
% #1 = name of the plane
% #2*x + #3*y + #4*z = #5 is the equation of the plane
\newcommand*\definePlaneByEquation[5]{
	\expandafter\gdef\csname tsx@plane@#1\endcsname{
		\def\tsx@plane@xcoeff{#2}
		\def\tsx@plane@ycoeff{#3}
		\def\tsx@plane@zcoeff{#4}
		\def\tsx@plane@scalar{#5}
	}
}

% Draw a plane.
% The optional first argument is passed as options to TikZ.
% The mandatory second argument is the name of the plane.
\newcommand\drawPlane[2][]{
	\tikzset{plane max x/.get=\tsx@plane@maxx}
	\tikzset{plane max y/.get=\tsx@plane@maxy}
	\tikzset{plane max z/.get=\tsx@plane@maxz}
	\csname tsx@plane@#2\endcsname
	
	\ifdim\tsx@plane@xcoeff pt=0pt
	\ifdim\tsx@plane@ycoeff pt=0pt
	\ifdim\tsx@plane@zcoeff pt=0pt
	%invalid plane
	\else % x=0, y=0
	\filldraw[plane,#1,shift={(0,0,\tsx@plane@scalar/\tsx@plane@zcoeff)}]
	(0,0,0) --
	(\tsx@plane@maxx,0,0) --
	(\tsx@plane@maxx,\tsx@plane@maxy,0) --
	(0,\tsx@plane@maxy,0) --
	cycle;
	\fi
	\else % x=0, y != 0
	\ifdim\tsx@plane@zcoeff pt=0pt % x=0, z=0
	\filldraw[plane,#1,shift={(0,\tsx@plane@scalar/\tsx@plane@ycoeff,0)}]
	(0,0,0) --
	(\tsx@plane@maxx,0,0) --
	(\tsx@plane@maxx,0,\tsx@plane@maxz) --
	(0,0,\tsx@plane@maxz) --
	cycle;
	\else % x=0
	\filldraw[plane,#1]
	(0,\tsx@plane@scalar/\tsx@plane@ycoeff,0) --
	(0,0,\tsx@plane@scalar/\tsx@plane@zcoeff) --
	(\tsx@plane@maxx,0,\tsx@plane@scalar/\tsx@plane@zcoeff) --
	(\tsx@plane@maxx,\tsx@plane@scalar/\tsx@plane@ycoeff,0) --
	cycle;
	\fi
	\fi
	\else % x!=0
	\ifdim\tsx@plane@ycoeff pt=0pt % x!=0,y=0
	\ifdim\tsx@plane@zcoeff pt=0pt % x!=0,y=0,z=0
	\filldraw[plane,#1,shift={(\tsx@plane@scalar/\tsx@plane@xcoeff,0,0)}]
	(0,0,0) --
	(0,0,\tsx@plane@maxz) --
	(0,\tsx@plane@maxy,\tsx@plane@maxz) --
	(0,\tsx@plane@maxy,0) --
	cycle;
	\else % x!=0,y=0,z!=0
	\filldraw[plane,#1]
	(\tsx@plane@scalar/\tsx@plane@xcoeff,0) --
	(0,0,\tsx@plane@scalar/\tsx@plane@zcoeff) --
	(0,\tsx@plane@maxy,\tsx@plane@scalar/\tsx@plane@zcoeff) --
	(\tsx@plane@scalar/\tsx@plane@xcoeff,\tsx@plane@maxy,0) --
	cycle;
	\fi
	\else % x!=0,y!=0
	\ifdim\tsx@plane@zcoeff pt=0pt % x!=0,y!=0,z=0
	\filldraw[plane,#1]
	(\tsx@plane@scalar/\tsx@plane@xcoeff,0) --
	(0,\tsx@plane@scalar/\tsx@plane@ycoeff,0) --
	(0,\tsx@plane@scalar/\tsx@plane@ycoeff,\tsx@plane@maxz) --
	(\tsx@plane@scalar/\tsx@plane@xcoeff,0,\tsx@plane@maxz) --
	cycle;
	\else % x!=0,y!=0,z!=0
	\filldraw[plane,#1]
	(\tsx@plane@scalar/\tsx@plane@xcoeff,0,0) --
	(0,\tsx@plane@scalar/\tsx@plane@ycoeff,0) --
	(0,0,\tsx@plane@scalar/\tsx@plane@zcoeff) --
	cycle;
	\fi
	\fi
	\fi
}

% Define a point.
% #1 = name of the point
% (#2,#3,#4) is the location.
% Also creates a coordinate node of name #1 at the location.
\newcommand\definePointByXYZ[4]{
	\coordinate (#1) at (#2,#3,#4);
	\expandafter\gdef\csname tsx@point@#1\endcsname{
		\def\tsx@point@x{#2}
		\def\tsx@point@y{#3}
		\def\tsx@point@z{#4}
	}
}

% Project a point to a plane.
% #1 = name of the new point
% #2 = name of old point
% #3 = name of plane
\newcommand\projectPointToPlane[3]{{
		\csname tsx@point@#2\endcsname
		\csname tsx@plane@#3\endcsname
		
		% square of norm of the normal vector
		\pgfmathparse{\tsx@plane@xcoeff*\tsx@plane@xcoeff + \tsx@plane@ycoeff*\tsx@plane@ycoeff + \tsx@plane@zcoeff*\tsx@plane@zcoeff}
		\let\nnormsq\pgfmathresult
		
		% Calculate distance in terms of the (non-normalized) normal vector
		\pgfmathparse{(\tsx@point@x*\tsx@plane@xcoeff + \tsx@point@y*\tsx@plane@ycoeff + \tsx@point@z*\tsx@plane@zcoeff - \tsx@plane@scalar) / \nnormsq}
		\let\distance\pgfmathresult
		
		% Calculate point
		\pgfmathparse{\tsx@point@x - \distance*\tsx@plane@xcoeff}
		\let\x\pgfmathresult
		\pgfmathparse{\tsx@point@y - \distance*\tsx@plane@ycoeff}
		\let\y\pgfmathresult
		\pgfmathparse{\tsx@point@z - \distance*\tsx@plane@zcoeff}
		\let\z\pgfmathresult
		
		\definePointByXYZ{#1}{\x}{\y}{\z}
	}}
	
	
\newcommand\projectPointToPlaneAlongZ[3]{{
	\csname tsx@point@#2\endcsname
	\csname tsx@plane@#3\endcsname

	\pgfmathparse{\tsx@point@x}	\let\x\pgfmathresult	
	\pgfmathparse{\tsx@point@y}	\let\y\pgfmathresult
	\pgfmathparse{ \tsx@plane@scalar - \tsx@point@x*\tsx@plane@xcoeff - \tsx@point@y*\tsx@plane@ycoeff}	\let\z\pgfmathresult
			
	\definePointByXYZ{#1}{\x}{\y}{\z}
	}}
	
\makeatother



\NewEnviron{eq}{
	\begin{align*}  % \begin{equation}\begin{split}	
	\BODY
	\end{align*} % \end{split}\end{equation}
}

\NewEnviron{eqRef}{
	\begin{align}
		\begin{split}
			\BODY
		\end{split}
	\end{align}
}



\renewcommand{\L}{\left(}
\newcommand{\R}{\right)}
\newcommand{\noi}{\noindent}
\newcommand{\thetaB}{\boldsymbol{\theta}}
\newcommand{\curvyL}{\mathcal{L}}
\newcommand{\lra}{\Leftrightarrow}
\newcommand{\uda}{\Updownarrow}
\newcommand{\ti}[1] {\textit{#1}}
\newcommand{\m}[1]{ \begin{bmatrix} #1 \end{bmatrix} }
\renewcommand{\det}[1]{ \begin{vmatrix} #1 \end{vmatrix} }
\renewcommand{\b}[1]{ \boldsymbol{#1} }
\newcommand{\tb}[1] {\textbf{#1}}
\renewcommand{\bf}[1]{ \textbf{#1} }
\newcommand{\eqs}{ \begin{align} }
\newcommand{\eqe}{ \end{align} }
%\newcommand{\be}[1]{ \begin{#1} }
%\newcommand{\l}{\left(}

%\renewcommand{\thesubsection}{\thesection.\alph{subsection}}  % use letters for subsections

\newcommand{\subsectionnote}[2]{ \subsection[#1]{#1\footnote{#2} } }
\newcommand{\wB}[0]{ \b{w} }
\newcommand{\xB}[0]{ \b{x} }
%\def\x \b{x}
\newcommand\gauss[2]{1/(#2*sqrt(2*pi))*exp(-((x-#1)^2)/(2*#2^2))} 
\newcommand\nGauss{\mathcal{N}}
\renewcommand{\t}[1]{  \text{#1}  }
\newcommand{\gaussFunctionText}[3]{ % 1=param/x, 2=mean/mu, 3=variance/sigma
	\frac{1}{ (#3)  \sqrt{2\pi}}  \exp \( - \frac {(#1- #2 )^2}{(2)  (#3) ^2}  \)
}
\newcommand{\hlineB}{  \Xhline{4\arrayrulewidth}   }  


\newcommand\plotfill[3]{
	\path[name path=#1Ground] (axis cs:#2,0) -- (axis cs:#3,0);  % just a horisontal line on the x-axis?
	\addplot [
	thick,
	color=blue,
	fill=blue, 
	fill opacity=0.1
	] 
	fill between[
	of=#1 and #1Ground,
	soft clip={domain=#2:#3},
	];	
}
\newcommand\ifrac[2]{\nicefrac{#1}{#2}}
\newcommand\half{{\nicefrac{1}{2}}}

\newcommand\tc[1]{\texttt{#1}}
\newcommand\calc[1]{\pgfmathparse{#1}\pgfmathresult}
\newcommand\round[2][2]{\pgfmathparse{#2}\pgfmathprintnumberto[precision=#1]{\pgfmathresult}{\skod}\skod }


\newcommand{\classificationError}[1]{ 
	1 - max #1
}
%\newcommand{\classification}[1]{ 
%	-	\sum_{k}^{k_{#1}}  {f_k}_{#1}\log_2 {f_k}_{#1} 
%}


\newcounter{myLoopCounter}
\newcommand{\repeatValue}[2][1]{%
	\forloop{myLoopCounter}{0}{\value{myLoopCounter}<#1}{#2}%
}

\newcommand{\note}[1]{   \parbox{15em}{#1}   }


%\def\norm#1{\left\|#1\right\|}
%\let\xB\b{x}
%\let\svvert\abe
%\let\Skod\left(

%\let\svverta\L(
%\let\svvertb\R)

\def\paranthesize#1{}

\def\i#1/#2{ \ifrac{#1}{#2}  }
\def\(#1\){\left(#1\right)}

\def\[#1\]{\left[#1\right]}

%\def\#1\\#2{\left[#1\right]}


%\def \'#1){\b{#1}}

\def\~#1{{ \widetilde{#1} }}


%\def\!#1\!{  \pgfmathparse{#1}\pgfmathresult  }


\pgfmathdeclarefunction{gauss}{2}{ 
	% #1=mean (mu), #2=variance (sigma)
	\pgfmathparse{1/(#2*sqrt(2*pi))*exp(-((x-#1)^2)/(2*#2^2))}%
}


%\tikzset{fontscale/.style = {font=\relsize{#1}}
%}

\newcommand\loadTabbedCsv[2]{ 
	\DTLsettabseparator
	\DTLloaddb{#1}{#2} 
	\DTLmaketabspace
	} 
	


\NewEnviron{loopCommaCsv}[3][]{{
%		\DTLsettabseparator
		%	\catcode`\^^I=12 %
		%	\DTLsetseparator{	}%		
		\ifthenelse{\DTLiseq{#1}{}}
		{}
		{	\DTLloaddb{#2}{#1}}%
		
		
		\newcommand{\rowNumber}{\arabic{DTLrowi}}
		\DTLforeach*{#2}{#3}{\BODY}
		%	\catcode`\^^I=12 %
		%	\DTLsetseparator{	}%		
		\DTLmaketabspace
	}}
	

\newcommand\foreachTabbedCsvRowInPlot[4][]{{
		\DTLsettabseparator
%		\ifthenelse{\DTLiseq{#1}{}}{}{ \DTLloaddb{#2}{#1}}	
		\newcommand{\rowNumber}{\arabic{DTLrowi}}
		\pgfplotsextra{
			\DTLforeach*{#2}{#3}{#4}
			}
		\DTLmaketabspace
}}




\NewEnviron{loopTabbedCsv}[3][]{{
		\DTLsettabseparator
		%	\catcode`\^^I=12 %
		%	\DTLsetseparator{	}%		
		\ifthenelse{\DTLiseq{#1}{}}
		{}
		{	\DTLloaddb{#2}{#1}} %
		
		
		\newcommand{\rowNumber}{\arabic{DTLrowi}}
		
		\DTLforeach*{#2}{#3}{\BODY}
		%	\catcode`\^^I=12 %
		%	\DTLsetseparator{	}%		
		\DTLmaketabspace
	}}

\NewEnviron{loopTabbedCsvAlt}[3]{{
	\DTLsettabseparator
	%	\catcode`\^^I=12 %
	%	\DTLsetseparator{	}%		

	\DTLloaddb{#2}{#1}
	
	
	\newcommand{\rowNumber}{\arabic{DTLrowi}}
	
	\DTLforeach*{#2}{#3}{\BODY}
	%	\catcode`\^^I=12 %
	%	\DTLsetseparator{	}%		
	\DTLmaketabspace
}}

\newcommand{\setSize}[1]{{\lvert #1 \rvert}}


\newcommand{\norm}[1]{{\lVert #1 \rVert}}
%\newcommand{\norm}[1]{{\lvert #1 \rvert}}

\newcommand{\mnorm}[1]{{\lVert #1 \rVert}}


\newcommand{\vectorsize}[1]{{\lvert #1 \rvert}}


%\lVert \mathbf{p} \rVert


\newcommand{\s}[1]{\mathbbm{#1}}



\newcommand\otherFont[1]{\mathfrak{#1}}

\newcommand\swirlR{\mathcal{R}}



%\renewcommand{\sqsubset}[1][0pt]{%
%	\mathrel{\raisebox{#1}{$\oldsqsubset$}}%
%}






\newcommand\superimpose[5]{
	\rlap{$#1$}{\hspace{#4}\raisebox{#5}{\scaleobj{#3}{#2}}}
}
\def\mytargetsymbol{\bullet}
%	\bigcirc
% \nearrow





\def\myskod{ \superimpose{\Box}{\mytargetsymbol}{0.9}{-0.1em}{0.45em} }

\def\cornerBL{	\scaleobj{0.77}{\superimpose{B}{L}{1}{0.485em}{0.34em}} }
\def\cornerBR{	\scaleobj{0.77}{\superimpose{B}{R}{1}{0.487em}{0.34em} } }
\def\cornerTL{	\scaleobj{0.77}{\superimpose{T}{L}{1}{0.45em}{0.32em} } }
\def\cornerTR{	\scaleobj{0.77}{\superimpose{T}{R}{1}{0.45em}{0.32em}} }


%
%
%\newenvironment{colfigure} {
%	\par\medskip\noindent\minipage{\linewidth}
%} {
%\endminipage\par\medskip
%}
%
%
%
%\newcommand{\myimage}[3][width=0.75\linewidth]{
%	\begin{colfigure}
%		\centering
%		\includegraphics[#1]{image/#2}
%		\captionof{figure}{\small #3}
%		\label{fig:#2}
%	\end{colfigure}
%}			
%\newcommand{\fullWidthDataFigure}[3][width=0.75\linewidth]{
%	\begin{figure*}
%		\centering
%		\includegraphics[#1]{/image/#2}
%		\captionof{figure}{\small #3}
%		\label{fig:#2}
%	\end{figure*}
%}			


%\usepackage[nottoc]{tocbibind}



\usepackage{etoolbox}
\patchcmd{\thebibliography}{\section*{\refname}}{}{}{}

% paragraph to subsubsubsection
\makeatletter
\renewcommand\paragraph{\@startsection{paragraph}{4}{\z@}%
	{-2.5ex\@plus -1ex \@minus -.25ex}%
	{1.25ex \@plus .25ex}%
	{\normalfont\normalsize\bfseries}}
\makeatother
\setcounter{secnumdepth}{3} % how many sectioning levels to assign numbers to
\setcounter{tocdepth}{3}    % how many sectioning levels to show in ToC

\newcommand\subsubsubsection[1]{\paragraph{\normalfont\fontsize{10.5pt}{1.2}\textsl{#1}} }














\newcommand\planeProjectPoint[5]{
	\definePointByXYZ{#2}{#3}{#4}{#5};
	\draw (#2) circle [radius=1pt];	
	\projectPointToPlane{proj#2}{#2}{#1}
	\fill (proj#2) circle [radius=1pt];
	\draw[->, shorten <=1pt,shorten >=1pt, color=gray] (#2) -- (proj#2);
}

\newcommand\planePointProjectExample[4]{
	\begin{tikzpicture}[x={(#1)}, y={(#2)}, z={(#3)},
	plane max z=3,#4]
	
	\node[text width=0.2cm,gray] at (4.1,-0.3,0) {\small x};
	\node[text width=0.2cm,gray] at (-0.3,4,0) {\small y};
	\node[text width=0.2cm,gray] at (-0.2,-0.3,4) {\small z};
	
	\draw[->] (0,0,0) -- (4,0,0);
	\draw[->] (0,0,0) -- (0,4,0);
	\draw[->] (0,0,0) -- (0,0,4);
	
	\definePlaneByEquation{myplane}{1}{1}{1}{3}
	\drawPlane[thick,fill=blue!10]{myplane}
	
	\planeProjectPoint{myplane}{p1}{1}{2}{1}
	\planeProjectPoint{myplane}{p3}{3}{2}{1}
	\planeProjectPoint{myplane}{p3}{3}{1}{1}
	\planeProjectPoint{myplane}{p3}{1}{1}{1}	
	
	
	\definePointByXYZ{topRight}{2.13}{1}{-0.13};	
	\definePointByXYZ{bottomRight}{2.19}{0.45}{0.36};
	\definePointByXYZ{topLeft}{0.81}{1.55}{0.64};
	\definePointByXYZ{bottomLeft}{0.87}{1}{1.13};

%	\draw[color=black] (topRight) circle [radius=1pt];		
%	\draw[color=black] (bottomRight) circle [radius=1pt];
%	\draw[color=black] (topLeft) circle [radius=1pt];		
%	\draw[color=black] (bottomLeft) circle [radius=1pt];		
%	
	
	\end{tikzpicture}
}


\newcommand\plotPlanarRectNoFit[5]{
	\begin{center}
		\begin{tikzpicture}
		\begin{axis}[ 
		%		every axis plot post/.append style={
		%			mark=none,domain=1:10 %,samples=50,smooth
		%		}, % All plots: from -2:2, 50 samples, smooth, no marks				
		axis x line=bottom,
		axis y line=left,
		xmin=#1, 
		xmax=#2, 
		ymin=#3,
		ymax=#4,
		legend pos=outer north east,
		xlabel={x},			
		ylabel={y},			
		y label style={
			rotate=-90,
			%						yshift=50.
			xshift=10
		},
		%		xscale=1.3,
		%		yscale=1.3,
		scale=0.9,
		]
	%		\plotClass{TOP}{\colorTop, thick}{transparent};
	%		\plotClass{RIGHT}{\colorRight,thick}{transparent};
	%		\plotClass{BOTTOM}{\colorBottom,thick}{transparent};
	%		\plotClass{LEFT}{\colorLeft,thick}{transparent};


		%		\plotClass{ORIGO}{gray}{gray, fill=white};
		
		\plotClass{INPUT_TOP}{\colorTopInput}{gray,fill=white};
		\plotClass{INPUT_RIGHT}{\colorRightInput}{gray,fill=white};
		\plotClass{INPUT_BOTTOM}{\colorBottomInput}{gray,fill=white};
		\plotClass{INPUT_LEFT}{\colorLeftInput}{gray,fill=white};
		\plotClass{INPUT_ORIGO}{transparent}{gray, fill=white};		
		%		\addlegendentry{$\gamma = 0/100$};
		%		\addlegendentry{$\gamma = 1/10$};
		%		\addlegendentry{$\gamma = 1$};
		%		\addlegendentry{$\gamma = 10$};
		\end{axis}
		\end{tikzpicture}
	\end{center}
}


\newcommand\plotPlanarRectFit[5]{
	\begin{center}
		\begin{tikzpicture}
		\begin{axis}[ 
		%		every axis plot post/.append style={
		%			mark=none,domain=1:10 %,samples=50,smooth
		%		}, % All plots: from -2:2, 50 samples, smooth, no marks				
		axis x line=bottom,
		axis y line=left,
		xmin=#1, 
		xmax=#2, 
		ymin=#3,
		ymax=#4,
		legend pos=outer north east,
		xlabel={x},			
		ylabel={y},			
		y label style={
			rotate=-90,
			%						yshift=50.
			xshift=10
		},
		%		xscale=1.3,
		%		yscale=1.3,
		scale=0.9,
		]
		\plotClass{TOP}{\colorTop, thick}{transparent};
		\plotClass{RIGHT}{\colorRight,thick}{transparent};
		\plotClass{BOTTOM}{\colorBottom,thick}{transparent};
		\plotClass{LEFT}{\colorLeft,thick}{transparent};
		%		\plotClass{ORIGO}{gray}{gray, fill=white};
		
		\plotClass{INPUT_TOP}{\colorTopInput}{gray!40,fill=white};
		\plotClass{INPUT_RIGHT}{\colorRightInput}{gray!40,fill=white};
		\plotClass{INPUT_BOTTOM}{\colorBottomInput}{gray!40,fill=white};
		\plotClass{INPUT_LEFT}{\colorLeftInput}{gray!40,fill=white};
		\plotClass{INPUT_ORIGO}{transparent}{gray!20, fill=white};		
		%		\addlegendentry{$\gamma = 0/100$};
		%		\addlegendentry{$\gamma = 1/10$};
		%		\addlegendentry{$\gamma = 1$};
		%		\addlegendentry{$\gamma = 10$};
		\end{axis}
		\end{tikzpicture}
	\end{center}
}



\newcommand\planeProjectSpatialExample[4][]{
	\begin{tikzpicture}[
	x={(#2)}, 
	y={(#3)}, 
	z={(#4)},
	plane max z=3
	,#1]
	
	\node[text width=0.2cm,gray] at (4,-0.3,0) {\small x};
	\node[text width=0.2cm,gray] at (-0.3,4,0) {\small y};
	\node[text width=0.2cm,gray] at (0,-0.25,4) {\small z};
	
	\draw[->] (0,0,0) -- (4,0,0);
	\draw[->] (0,0,0) -- (0,4,0);
	\draw[->] (0,0,0) -- (0,0,4);
	
	\definePlaneByEquation{myplane}{1}{1}{1}{3}
	\drawPlane[thick,fill=blue!10]{myplane}
	
	\planeProjectPoint{myplane}{p1}{1}{2}{1}
	\planeProjectPoint{myplane}{p3}{3}{2}{1}
	\planeProjectPoint{myplane}{p3}{3}{1}{1}
	\planeProjectPoint{myplane}{p3}{1}{1}{1}	
	
	
	%	\definePointByXYZ{Skod}{0.6666}{1.6666}{0.66666};
	\definePointByXYZ{fittedOrigo}{1.5}{1}{0.5};
	\draw[color=gray] (fittedOrigo) circle [radius=2pt];		
	
	\definePointByXYZ{topRight}{2.13}{1}{-0.13};
	\draw[color=green] (topRight) circle [radius=1pt];		
	
	\definePointByXYZ{bottomRight}{2.19}{0.45}{0.36};
	\draw[color=orange] (bottomRight) circle [radius=1pt];
	
	\definePointByXYZ{topLeft}{0.81}{1.55}{0.64};
	\draw[color=blue] (topLeft) circle [radius=1pt];		
	
	\definePointByXYZ{bottomLeft}{0.87}{1}{1.13};
	\draw[color=purple] (bottomLeft) circle [radius=1pt];		
	
	\draw[-, shorten <=1pt,shorten >=1pt, color=\colorRight, thick] (topRight) -- (bottomRight); % RIGHT
	\draw[-, shorten <=1pt,shorten >=1pt, color=\colorBottom, thick] (bottomRight) -- (bottomLeft); % BOTTOM
	\draw[-, shorten <=1pt,shorten >=1pt, color=\colorLeft, thick] (bottomLeft) -- (topLeft); % LEFT
	\draw[-, shorten <=1pt,shorten >=1pt, color=\colorTop, thick] (topLeft) -- (topRight); % TOP
	
	\end{tikzpicture}
}

\newcommand\planeProjectNoFit[4]{
	\begin{tikzpicture}[x={(#1)}, y={(#2)}, z={(#3)},
	plane max z=3,#4]
	
	\node[text width=0.2cm,gray] at (4,-0.3,0) {\small x};
	\node[text width=0.2cm,gray] at (-0.3,4,0) {\small y};
%	\node[text width=0.2cm,gray] at (0,-0.25,4) {\small z};
	
	\draw[->] (0,0,0) -- (4,0,0);
	\draw[->] (0,0,0) -- (0,4,0);
	\draw[->] (0,0,0) -- (0,0,4);
	
	\definePlaneByEquation{myplane}{1}{1}{1}{3}
	
	\drawPlane[thick,fill=blue!10]{myplane}
	
	\planeProjectPoint{myplane}{p1}{1}{2}{1}
	\planeProjectPoint{myplane}{p3}{3}{2}{1}
	\planeProjectPoint{myplane}{p3}{3}{1}{1}
	\planeProjectPoint{myplane}{p3}{1}{1}{1}	
	
	
	%	\definePointByXYZ{Skod}{0.6666}{1.6666}{0.66666};
	\definePointByXYZ{fittedOrigo}{1.5}{1}{0.5};
	\draw[color=gray] (fittedOrigo) circle [radius=2pt];		
	
%	\definePointByXYZ{topRight}{2.13}{1}{-0.13};
%	\draw[color=green] (topRight) circle [radius=1pt];		
%	
%	\definePointByXYZ{bottomRight}{2.19}{0.45}{0.36};
%	\draw[color=orange] (bottomRight) circle [radius=1pt];
%	
%	\definePointByXYZ{topLeft}{0.81}{1.55}{0.64};
%	\draw[color=blue] (topLeft) circle [radius=1pt];		
%	
%	\definePointByXYZ{bottomLeft}{0.87}{1}{1.13};
%	\draw[color=purple] (bottomLeft) circle [radius=1pt];		
	
%	\draw[-, shorten <=1pt,shorten >=1pt, color=\colorRight, thick] (topRight) -- (bottomRight); % RIGHT
%	\draw[-, shorten <=1pt,shorten >=1pt, color=\colorBottom, thick] (bottomRight) -- (bottomLeft); % BOTTOM
%	\draw[-, shorten <=1pt,shorten >=1pt, color=\colorLeft, thick] (bottomLeft) -- (topLeft); % LEFT
%	\draw[-, shorten <=1pt,shorten >=1pt, color=\colorTop, thick] (topLeft) -- (topRight); % TOP
	
	\end{tikzpicture}
}

\def\plotXMin{\pgfkeysvalueof{/pgfplots/xmin}}
\def\plotXMax{\pgfkeysvalueof{/pgfplots/xmax}}

\def\plotYMin{\pgfkeysvalueof{/pgfplots/xmin}}
\def\plotYMax{\pgfkeysvalueof{/pgfplots/ymax}}

\newcommand\drawXAxis[1]{
	\draw[#1] (axis cs:\pgfkeysvalueof{/pgfplots/xmin},0) -- (axis cs:\pgfkeysvalueof{/pgfplots/xmax},0);	
}

\newcommand\drawYAxis[1]{
	\draw[#1] (axis cs:0,\pgfkeysvalueof{/pgfplots/ymin}) -- (axis cs:0,\pgfkeysvalueof{/pgfplots/ymax});
}


\def\drawFunctionX{
	\draw[gray!50, dashed] plot (axis cs:\plotXMin,\plotXMin) -- (axis cs:\plotXMax,\plotXMax);
	}

\makeatletter
\pgfplotstableset{
	discard if not/.style 2 args={
%		row predicate/.append code={		
		row predicate/.code={
			\def\pgfplotstable@loc@TMPd{\pgfplotstablegetelem{##1}{#1}\of}
			\expandafter\pgfplotstable@loc@TMPd\pgfplotstablename
			\edef\tempa{\pgfplotsretval}
			\edef\tempb{#2}
			\ifx\tempa\tempb
			\else
			\pgfplotstableuserowfalse
			\fi
		}
	}
}
\makeatother

\newcommand\plotOptionRotateAroundOrigin[1]{rotate around={#1:(axis cs:0,0)}}

\tikzset{
	error band/.style={fill=orange},
	error band style/.style={
		error band/.append style=#1
	}
}


%\newcommand{\expScatterPlotClass}[5]{
%	\addplot[
%%		error band,
%		%			visualization depends on=\thisrow{alignment} \as \alignment,
%		%			every node near coord/.style={anchor=\alignment} ,
%		%			nodes near coords*={(5,5)},
%		%			nodes near coords*={$(\pgfmathprintnumber[frac]\myvalue)$},
%		%			visualization depends on={\thisrow{targetValue} \as \myvalue}
%		%			clickable coords={\thisrow{label}}, 
%		%		color=red,
%		mark size=0.5mm,
%		enlarge x limits=true,
%		enlarge y limits=true,
%		scatter,
%		color=\meanColor,
%		%			scatter src=explicit symbolic, 
%		point meta=explicit symbolic,
%		%		point meta=explicit symbolic,
%		scatter/classes={
%%			lily={mark=*,\myRed,line width=0mm},
%%			leah={mark=*,\myGreen,line width=0mm},		
%%%			evan={mark=*,orange,line width=0mm},		
%%%			rose={mark=*,pink,line width=0mm},		
%%			mean={mark=none,\meanColor,line width=0mm},
%%%			std{mark=none,azure,line width=0mm},
%%			unnamed={mark=*,black,line width=0mm}		
%			lily=\lorot,
%			leah={mark=*,\myGreen,line width=0mm},
%			evan={mark=*,\myBlue,line width=0mm},		
%			rose={mark=*,violet,line width=0mm},		
%			unnamed={mark=*,pink,line width=0mm},		
%			mean={mark=none,\meanColor,line width=0mm},
%			std={mark=none,brown,line width=0mm}
%		},#5
%		] 	
%		plot[error bars/.cd, y dir = both, y explicit]
%		table[
%		meta=sessionName,
%		%			  col sep = \csvSep,
%		x=#2,	%trialTargetValue,
%		%			  y=trialTargetValue,
%		%			  y=trialReleasePositionTargetDiff
%		y=#3,
%		y error=#4			
%		%			  ] {\datafile};
%	] {#1};
%	
%%	  \addplot [draw=none, stack plots=y, forget plot, x=#2, y=#3, #5] {#1};
%%	  \addplot +[draw=none, stack plots=y, error band, x=#2, y=#3, #5] {#1};
%%	  \addplot [draw=none, stack plots=y, forget plot, x=#2, y=#3, #5] {#1};	  
%%	  \addplot [forget plot] {#1};
%	  
%	
%	%% Lower bound (invisible plot)
%	%\addplot [draw=none, stack plots=y, forget plot] table [
%	%x=#2,
%	%y expr=\thisrow{#3}-\thisrow{#4}, #5
%	%] {#1};
%	%
%	%% Stack twice the error, draw as area plot
%	%\addplot [draw=none, fill=gray!40, stack plots=y, area legend, ] table [
%	%x={#2},
%	%y expr=2*\thisrow{#4}, #5
%	%] {#1} \closedcycle;
%	%
%	%% Reset stack using invisible plot
%	%\addplot [forget plot, stack plots=y,draw=none, #5] table [x={#2}, y expr=-(\thisrow{#3}+\thisrow{#4})] {#1};
%	
%}

%\usepackage{subfig}
\usepackage{caption}

\def\true{true}
\def\plotMean{\edef\isMeanToBePlotted{\true}}


\def\perpAxesOptions{
%	ytick pos=left,
	xlabel near ticks,
	y label style={ 
		rotate=180, 
		xscale=-1, 
%		yscale=-1,  
	},
	x label style={ 
		rotate=180, 
		xscale=-1, 
%		yscale=-1 
	},
	yticklabel style={ 
		rotate=-90, 
		xscale=-1  
	},
%	xtick pos=right,
	xticklabel style={ 
		rotate=-90, 
		xscale=-1 
	},
}


\newcommand\expScatterPlotPerp[2]{
	\expScatterPlot{#1}{#2}{rotate=90,yscale=-1}{\perpAxesOptions}
}



\newcommand\drawAxes{
		\drawYAxis{ultra thin,gray!30}
		\drawXAxis{<->, thick, gray!50}
	}

\pgfplotsset{minor grid style={dashed,red}}
\pgfplotsset{major grid style={dotted,green!50!black!30}}



%
%\newcommand\GetMean[2]{
%	\pgfplotstableread[col sep = \csvSep]{#1}\tableA
%	\pgfplotstableset{
%		create on use/new/.style={
%%			discard if not={targetValue}{-8},
%			create col/expr={
%					\pgfmathaccuma + \thisrow{#2}
%				},
%			},
%		}
%	\pgfplotstablegetrowsof{\tableA}
%	\pgfmathsetmacro{\NumRows}{\numexpr\pgfplotsretval-1\relax}
%
%%	\pgfplotstablegetelem{\numexpr\NumRows-1\relax}{new}\of{#1} 
%%	\pgfplotstablegetelem{\NumRows}{new}\of{#1} 
%	\pgfmathsetmacro{\Sum}{\pgfplotsretval}
%	\pgfmathsetmacro{\Mean}{\Sum/\NumRows}
%}
%
%
%\newcommand\DrawVMean[1][]{
%	\draw[#1] 
%	(axis cs:\Mean,\pgfkeysvalueof{/pgfplots/ymin}) -- 
%	(axis cs:\Mean,\pgfkeysvalueof{/pgfplots/ymax});
%}
%\newcommand\DrawHMean[1][]{
%	\draw[#1] 
%	(axis cs:\pgfkeysvalueof{/pgfplots/xmin},\Mean) -- 
%	(axis cs:\pgfkeysvalueof{/pgfplots/xmax},\Mean);
%}
%
%
%
%
%
%
%

%%%% EXP1



%\newcommand\expScatterPlotNoStats{
%	\expScatterPlotClass\expOneCsvFilepath\plotX\plotY{}{
%		only marks,
%		discard if={sessionName}{mean},
%		discard if={sessionName}{std},
%%		discard if={sessionName}{errorMargin},
%		discard if not={trialName}{\trialName},
%	}	
%}

\newcommand\expScatterPlotMean[2][\meanColor]{
%	\edef\plotMeanColor{#1}
	\expScatterPlotClass\expOneCsvFilepath\plotX\plotY{#2}{
		discard if not={sessionName}{mean},
		discard if not={trialGroupName}\trialGroupName,
		color=#1
	}	
}


\newcommand{\expOneAxis}[2][]{	
	\begin{axis}[ 
		xmajorgrids=true, ymajorgrids=true, 
		ymin=0,   
%		ymax=\baselinePlotYMax,
		xmin=-55, xmax=55,
		%		nodes near coords,
		xlabel={Target value},
		ylabel={Duration (seconds)},
		xtick distance={10},#1
		]
		#2
	\end{axis}			
	}


\newcommand{\expOneGroupPlot}[2][]{
	\nextgroupplot[ 
		xmajorgrids=true, ymajorgrids=true, 
		ymin=0,   
		ymax=\baselinePlotYMax,
		xmin=-55, xmax=55,
		%		nodes near coords,
		xlabel={Target value},
		ylabel={Duration (seconds)},
		xtick distance={10},#1
		]
	#2
}


\newcommand\expOnePlot[2]{
		\addplot[		
		%			domain=-50:50,
		%		error band,
		%			visualization depends on=\thisrow{alignment} \as \alignment,
		%			every node near coord/.style={anchor=\alignment} ,
		%			nodes near coords*={(5,5)},
		%			nodes near coords*={$(\pgfmathprintnumber[frac]\myvalue)$},
		%			visualization depends on={\thisrow{targetValue} \as \myvalue}
		%			clickable coords={\thisrow{label}}, 
		%		color=red,
		mark size=0.5mm,
		enlarge x limits=true,
		enlarge y limits=true,
		scatter,
		color=\meanColor,
		%			scatter src=explicit symbolic, 
		point meta=explicit symbolic,
		%		point meta=explicit symbolic,
		scatter/classes={
				lily={mark=*,\lilyColor,line width=0mm},
				Lily={mark=*,\lilyColor,line width=0mm},
			leah={mark=*,\leahColor,line width=0mm,xshift=-0.05cm},
				evan={mark=*,\evanColor,line width=0mm,xshift=0.05cm},		
				evann={mark=*,\evanColor,line width=0mm,xshift=0.05cm},		
			rose={mark=*,\roseColor,line width=0mm,xshift=0.1cm},		
			rosee={mark=*,\roseColor,line width=0mm,xshift=0.1cm},
				sonia={mark=*,\soniaColor,line width=0mm,xshift=-0.1cm},
				soniaa={mark=*,\soniaColor,line width=0mm,xshift=-0.1cm},
			mike={mark=*,\mikeColor,line width=0mm,xshift=-.15cm},
			mikee={mark=*,\mikeColor,line width=0mm,xshift=-.15cm},
			unnamed={mark=*,\unnamedColor,line width=0mm,xshift=0.15cm},		
			mean={mark=none,\meanColor,line width=0mm},
%			mean={mark=none,\meanColor,line width=0mm},
				std={mark=none,\stdColor,line width=0mm}
		},
		%			unbounded coords=skip
		,#1
		] 	
		plot[error bars/.cd, y dir = both, y explicit]
		table[
		meta=sessionName,
		%			  col sep = \csvSep,
		%		x=#2,	%trialTargetValue,
		%		%			  y=trialTargetValue,
		%		%			  y=trialReleasePositionTargetDiff
		%		y=#3,
		%		y error=#4	
%		#2] {\dataSource}; %\expOneCsvFilepath};
		#2] {\expOneCsvFilepath};
}






\newcommand\expOneAxisPlot[3][]{
	\expOneAxis[#1]{
		\expOnePlot{#2}{#3}
		\drawAxes						
%		#1	
	}
}

%\newenvironment{\expOnePlot}[2][]{\expOneAxis{#1}}{}

%\def\filterNoStats{}

\newcommand\expOneScatterPlot[5]{
	\begin{axis}[ 
		xmajorgrids=true, ymajorgrids=true, 
		ymin=0,   ymax=\baselinePlotYMax,
		xmin=-55, xmax=55,
%		domain=-50:50,
			%		nodes near coords,
		xlabel={Target value},
		ylabel={Duration (seconds)},
		xtick distance={10},
%		unbounded coords=skip,
		,#5
		]
		
		\drawAxes				
		
		\edef\plotX{#2} %{trialTargetValue}
		\edef\plotY{#3}%{trialDurationSeconds}
		\edef\trialGroupName{#1}
%		\edef\errorCol{#1}
		%		

		\expScatterPlotClass\expOneCsvFilepath\plotX\plotY{}{
			only marks,
%			discard if={sessionName}{mean},
%			discard if={sessionName}{std},
			discard if={trialIsStatistic}{1},
			%		discard if={sessionName}{errorMargin},
			discard if not={trialGroupName}{\trialGroupName},
		}	
		#4
		
%		\edef\plotErrorMarginColName{#4}%{trialDurationSecondsConfidence}


%		\isMeanPlotted
		
%		\expScatterPlotMean{}
%		\ifthenelse{\equal{\plotErrorMarginColName}{}}{	}{ \expScatterPlotMean	} 


%		\else
%%					\expScatterPlotMean
%%			\expScatterPlotClass{#1}{#2}{#3}{#5, discard if not={sessionName}{mean}}
%		\fi
		
	\end{axis}			
}




%%%%%%%%





%\pgfplotstablesort[sort key={T}]{\sorted}{\data} %get the data and sort by column 'T'

%\newcommand\filter[2]{	discard if={#1}{#2} }



\newcommand\threeFigure[3]{
	%	\hspace*{-1.8cm}
%	\def\threeFigureTikzScale{1.1}%{0.83}
	\begin{minipage}[b]{0.3\textwidth} #1 \end{minipage}
	\begin{minipage}[b]{0.3\textwidth} #2 \end{minipage}
	\begin{minipage}[b]{0.3\textwidth} #3 \end{minipage}\\
}

\newcommand\twoFigure[2]{
	%	\hspace*{-1.8cm}
	\begin{minipage}[b]{0.45\textwidth} #1 \end{minipage}
	\begin{minipage}[b]{0.45\textwidth} #2 \end{minipage}\\
}




\newcommand\threeFigureTrialMode[5][]{ % groupName, projection mode
	\def\noYTickLabels{{,,}}
	\def\yLabelWhite{\textcolor{white}{}}
	%	\def\groupName{\IfStrEq{#1}{}{#2}{#1}}	
%	\hspace{-2cm}
%	\threeFigure{
%%%%%%%%%%%%%%%%%%%%%%%%%
	\begin{groupplot}[
%		width=0.5\textwidth,
		group style={
			group size=2 by 3,
			x descriptions at=edge bottom,
			y descriptions at=edge left,
			horizontal sep=0,
			vertical sep=0.4cm
		}
		]

	%	\nextgroupplot[y post scale=3]
	%	\addplot {2*x+5};
	%	\addplot {3*x+5};
	
	%	\nextgroupplot[y post scale=3]
	%	\addplot {2*x+5};
	
		\threeFigureTrialModeAxisPlotWithMean[
			,#4, 
			legend entries={
				mean trial duration (seconds)			
			},	
			]{Horizontal}{#3}{#2-Horisontal}
			%				{PlaneNormal-#2}
	
		  \coordinate (top) at (rel axis cs:0,1);% coordinate at top of the first plot
		  
		\trialAndProjectionModeTikz[
			,#5, 
			legend entries={
	%			mean duration (seconds),			
				overshoot travel (\%),
				inertia travel (\%),
				touch count $\times$ 10
			},	
%			every extra y tick/.style={
%				grid=none, 
%%				tick0/.initial=red,
%%				tick1/.initial=green,
%%				tick2/.initial=orange,
%				yticklabel style={
%%					anchor=north, 
%%					color=\pgfkeysvalueof{/pgfplots/tick\ticknum},
%				},
%			},
			]
			{Horizontal}{#3}{#2-Horisontal}	
	
	
		\threeFigureTrialModeAxisPlotWithMean[
			,#4,
			]{Vertical}{#3}{#2-Vertical}
			%				{PlaneNormal-#2}
		\trialAndProjectionModeTikz[
			,#5, 
	%		ylabel={}%\empty, % xtick=\empty, xlabel=\empty,
			]
		{Vertical}{#3}{#2-Vertical}	
		
		\threeFigureTrialModeAxisPlotWithMean[
			,#4,
			]{Combined}{#3}{#2-DimsConcatenationAveraged}
			%				{PlaneNormal-#2}
		\trialAndProjectionModeTikz[
			,#5, 
	%		ylabel={}%\empty, % xtick=\empty, xlabel=\empty,
			]
			{Combined}{#3}{#2-DimsConcatenationAveraged}	
		
		
		\coordinate (bot) at (rel axis cs:1,0);% coordinate at bottom of the last plot
	

	\end{groupplot}
	
	\path (top-|current bounding box.west)-- 
		node[anchor=south,rotate=90] {\large Duration (seconds)} 
		(bot-|current bounding box.west);
	
	\path (top-|current bounding box.east)-- 
	node[anchor=south,rotate=-90] {\large Secondary statistics} 
	(bot-|current bounding box.east);			

	\node[anchor=south] at ($(current bounding box.south) - (0,0.8)$) {Target value / Distance (cm)}; 
}

\usepackage{colordvi}


\newcommand\expOnePlotMean[3][\meanColor]{ % [1]: plot options, 2: trialGroupName, 3: table options  
	\expOnePlot{
		#1,
		filter in={sessionName}{mean},
		filter in={trialGroupName}{#2}, %\filter,		%Baseline-Horisontal		Baseline-DimsConcatenationAveraged					
	}{x=trialTargetValue, y=trialDurationSeconds, y error=trialDurationSecondsConfidence,#3}
}

\newcommand\trialAndProjectionModeTikz[4][]{  % [1]: axis 
%	\begin{tikzpicture}[scale=1] 
	\expOneGroupPlot[
		legend style={draw=none,
%			fill=none,
			font=\scriptsize\selectfont,
			at={(axis cs:0,\plotYMax*.9)},anchor=north
			},
		xlabel=,
		ylabel=,
%		xlabel={Target value}, 
%		legend style={
			%			empty legend, 
			%			draw=none, 
%			at={(axis cs:0,\plotYMax-3)},anchor=north
%		},
%		legend entries={
%%			mean duration (seconds),			
%			overshoot travel (\%),
%			inertia travel (\%),
%			touch count
%		},	
		extra y ticks ={0,10,...,100},
%		extra y tick labels={0\%,10\%,...\%,100\%},
		extra y tick labels={0,10,...,100},
		%			extra ytick scale label code/.code={$\cdot \pi$},
		%			extra y tick labels={$s_l$,$s_r$,qr},
		yticklabel pos=right, 
		ylabel near ticks , % axis y line*=right, ,ylabel near ticks 
		%			ylabel={}%\empty, % xtick=\empty, xlabel=\empty,
		,#1,
		]{  %south west]{

		%		\addlegendentry[gray]{\text{\large #2}}		
		\addlegendimage{thick,no markers, dashed, \trialOvershootTravelRatioColor}
		\addlegendimage{thick,no markers,dashed, \trialInertialDistanceTraveledRatioColor}
		\addlegendimage{thick,no markers,dashed,\trialTouchCountColor}
		
		%				\def \filter{ }	
		\expOnePlot{
			%			yscale=0.1,
			smooth,
			yscale=10,
			,#3,
%			ybar,
%			bar width=0.3cm,	
%			fill=\trialTouchCountColor,
			color=\trialTouchCountColor,
%			dashed,
			%			only marks,
			%			filter out={trialIsStatistic}{1},
			filter in={trialGroupName}{#4},
			filter in={sessionName}{mean},
		}{x=trialTargetValue, y=trialTouchCount}
		
		\expOnePlot{
			yscale=100,
			%			yellow,
			%			yscale=2,
			smooth,
			#3,
			color=\trialOvershootTravelRatioColor,
%			dashed,
			%			only marks,
			%			filter out={trialIsStatistic}{1},
			filter in={trialGroupName}{#4},
			filter in={sessionName}{mean},
		}{x=trialTargetValue, y=trialOvershootTravelRatio}
		
		\expOnePlot{
			yscale=100,
			%			yellow,
			%			yscale=2,
			smooth,
			#3,
			color=\trialInertialDistanceTraveledRatioColor,
%			dashed,
			%			only marks,
			%			filter out={trialIsStatistic}{1},
			filter in={trialGroupName}{#4},
			filter in={sessionName}{mean},
		}{x=trialTargetValue, y=trialInertialDistanceTraveledRatio}
		
		\draw[lightgray,densely dotted] (axis cs:0,0) -- (axis cs:0,\pgfkeysvalueof{/pgfplots/ymax});			
	}	
%	\end{tikzpicture}\\	
}

\newcommand\threeFigureTrialModeAxisPlotWithMean[4][]{  % [1]: axis options, 2: legend text, 3: mean and non-mean plot options, 4: trialGroupName
%	\begin{tikzpicture}[scale=1] 
		\expOneGroupPlot[%\expOneAxis[
%			xlabel={Target value}, 
			xlabel=,
			ylabel={#2},
%			legend style={
%				empty legend, 
%				draw=none, 
%				at={(axis cs:0,\plotYMax-3)},anchor=north
%				},
		legend style={
			draw=none,
			%			fill=none,
			at={(axis cs:0,\plotYMax*.9)},
			anchor=north,
			font=\scriptsize\selectfont,
			%			font=\fontsize{4}{5}\selectfont,
		},
%		y ticks ={0,10,...,100},
%			ytick={0\%,10\%,...\%,100\%},
			yticklabels={,,},
			extra y ticks ={0,10,...,100},			
			extra y tick labels={0,10,...,100},
%			extra y tick labels={0s,10s,...s,100s},
			,#1,
			]{  %south west]{
%			\drawAxes		

			\addlegendimage{thick,no markers, \meanColor}
			
			%				\def \filter{ }	
			\expOnePlotMean[smooth,thick,#3,]{#4}{}
			\expOnePlot{
				#3,
				only marks,
				filter out={trialIsStatistic}{1},
				filter in={trialGroupName}{#4},		
			}{x=trialTargetValue, y=trialDurationSeconds}
			\draw[lightgray,densely dotted] (axis cs:0,0) -- (axis cs:0,\pgfkeysvalueof{/pgfplots/ymax});	
		}	
%	\end{tikzpicture}\\	
}


\newcommand\readCsvFiltered[3]{
	\def\tmpFilename{test.dat}
	%\pgfplotstableread[
	\pgfplotstabletypeset[
	string type,
	%every head row/.style={output empty row},
	%output empty row,
	begin table={},
	end table={},
	%	begin table/.add={}{},
	%	begin table={},
	%		end table={},
	outfile = \tmpFilename,
	skip coltypes=true,
	%write to macro=\total,
	typeset=false,
	%col sep=ampersand,row sep=\\,
	%postproc cell content/.append style={},
	%postproc cell content/.append style={},
	%	begin table={\begin{verbatim}},
	%	end table={\end{verbatim}},
	every head row/.style={},
	% every last row/.style={},
	% col sep= comma,	% specify the column separation character
	% row sep= newline,
	TeX comment={},
	typeset cell/.code={
		\ifnum\pgfplotstablecol=\pgfplotstablecols
		\pgfkeyssetvalue{/pgfplots/table/@cell content}{##1}
		\else
		\pgfkeyssetvalue{/pgfplots/table/@cell content}{##1	}
		\fi
	},#2,
	%	columns={sessionName, trialFinalOffscreenX, trialFinalOffscreenY,  trialFinalOffscreenZ},
	% columns/sessionName/.style={string type},		
	]{#1}%{\dataSource}	
	
	\pgfplotstableread{\tmpFilename}{#3}
}


\def\makeTicksFontScriptSize{\pgfplotsset{every tick label/.append style={font=\scriptsize}}}






%\begin{filecontents*}{testdata.csv}
%	xc,yc,xer,yer,phi
%	1,4,0.04,0.02,0.5
%	2,3,0.87,0.24,1
%	3,5,0.02,0.3,2.35
%	4,1,0.4,0.9,2.5
%	5,3,0.2,0.1,0.2
%	2.5,3,1.2,0.5,0.2
%	3,2,1,0.25,2.3
%\end{filecontents*}






\newcommand\plotPolarTargetValues{
	\addplot[
		scatter,				only marks,		
		point meta=explicit symbolic,
		line width=0mm,
		mark size=1.8,
		scatter/classes={
			10=\targetColor{10}!80!black,
			20=\targetColor{20}!80!black,
			30=\targetColor{30}!80!black,
			40=\targetColor{40}!80!black,
			50=\targetColor{50}!80!black,
			-10=\targetColor{-10}!80!black,
			-20=\targetColor{-20}!80!black,
			-30=\targetColor{-30}!80!black,
			-40=\targetColor{-40}!80!black,
			-50=\targetColor{-50}!80!black
	%		mean={mark=*,#3,line width=0.5mm},
%			target={mark=*,#3!80!black,line width=0mm,mark size=1}%%%%%%%%%%%%%%%%%%%%%%%%%%%%%%%%		
		}						
		]%surf, mesh/rows=10] 	
		plot[->,error bars/.cd, y dir = both, y explicit]
		table[
		x=trialReleasePositionX,
		y=trialReleasePositionY,
		meta=trialTargetValue,
		filter in={sessionName}{target}
		] 
		{data/exp1/TrialSesssions-(E2-mean,target).csv};
}



\newcommand\expPolarPlot[4][]{
	\addplot[
		shorten >=.05cm,
		shorten <=.05cm,
		scatter,		
		point meta=explicit symbolic,
		#4!50!black,
		%			red,
		scatter/classes={
			mean={mark=*,#4,line width=0.5mm,opacity=1},
			target={mark=*,#4!80!black,line width=0mm,mark size=1}%%%%%%%%%%%%%%%%%%%%%%%%%%%%%%%%		
		}						
		]%surf, mesh/rows=10] 	
		plot[->,error bars/.cd, y dir = both, y explicit]
		table[
		x=trialReleasePositionX,
		y=trialReleasePositionY,
		meta=sessionName,#1,
		%		filter multiple=sessionName equals mean or target,
		%		filter in={trialUIAngleDegrees}{#1},
		%		filter in={trialTargetValue}{#2}
		] 
		%			{data/exp1/TrialSesssions-(E2-polar-evan,rose,Lily,Lily).csv} 
		{data/exp1/TrialSesssions-(E2-mean,target-trialTargetValue(#3),trialUIAngleDegrees(#2)).csv};
		%		{\expTwoPolarCsvFilepath};	
}


\newcommand\expPolarPlotAngleExpThree[3][]{
	\expPolarPlot{#2}{-10}{#3, target-10#1}
	\expPolarPlot{#2}{-20}{#3, target-20#1}
	\expPolarPlot{#2}{-30}{#3, target-30#1}
	\expPolarPlot{#2}{-40}{#3, target-40#1}
	\expPolarPlot{#2}{-50}{#3, target-50#1}
	\expPolarPlot{#2}{10}{#3, target10#1}
	\expPolarPlot{#2}{20}{#3, target20#1}
	\expPolarPlot{#2}{30}{#3, target30#1}
	\expPolarPlot{#2}{40}{#3, target40#1}
	\expPolarPlot{#2}{50}{#3, target50#1}
}


\newcommand\expPolarPlotAngle[2][]{
	\expPolarPlot{#2}{-10}{target-10#1}
	\expPolarPlot{#2}{-20}{target-20#1}
	\expPolarPlot{#2}{-30}{target-30#1}
	\expPolarPlot{#2}{-40}{target-40#1}
	\expPolarPlot{#2}{-50}{target-50#1}
	\expPolarPlot{#2}{10}{target10#1}
	\expPolarPlot{#2}{20}{target20#1}
	\expPolarPlot{#2}{30}{target30#1}
	\expPolarPlot{#2}{40}{target40#1}
	\expPolarPlot{#2}{50}{target50#1}
}

\newcommand\expPolarPlotAngleAlt[1]{
	\expPolarPlot{#1}{-10}{angle#1}
	\expPolarPlot{#1}{-20}{angle#1}
	\expPolarPlot{#1}{-30}{angle#1}
	\expPolarPlot{#1}{-40}{angle#1}
	\expPolarPlot{#1}{-50}{angle#1}
	\expPolarPlot{#1}{10}{angle#1}
	\expPolarPlot{#1}{20}{angle#1}
	\expPolarPlot{#1}{30}{angle#1}
	\expPolarPlot{#1}{40}{angle#1}
	\expPolarPlot{#1}{50}{angle#1}
}

%\def\defaultRadialStyle{}

\newcommand\polarPlotDrawRadials[1][lightgray, dashed, shorten >=.03cm, shorten <=.12cm]{
	\draw[#1] (origin) -- +(0:\radius) node[right]{0};
	\draw[#1] (origin) -- +(30:\radius) node[above right]{30};
	\draw[#1] (origin) -- +(60:\radius) node[above right]{60};
	\draw[#1] (origin) -- +(90:\radius) node[above]{90};
	\draw[#1] (origin) -- +(120:\radius) node[above left]{120};
	\draw[#1] (origin) -- +(150:\radius) node[above left]{150};
	\draw[#1] (origin) -- +(180:\radius) node[left]{180};
	\draw[#1] (origin) -- +(210:\radius) node[left]{210};
	\draw[#1] (origin) -- +(240:\radius) node[below left]{240};
	\draw[#1] (origin) -- +(270:\radius) node[below]{270};
	\draw[#1] (origin) -- +(300:\radius) node[below right]{300};
	\draw[#1] (origin) -- +(330:\radius) node[below right]{330};
}

\newcommand\polarPlotDrawRadialsColorByAngle[1][dashed, shorten >=.03cm, shorten <=.12cm]{
	\draw[#1, \angleColor{0}!80!black] (origin) -- +(0:\radius) node[right]{0};
	\draw[#1, \angleColor{30}!80!black!70] (origin) -- +(30:\radius) node[above right]{30};
	\draw[#1, \angleColor{60}!80!black!70] (origin) -- +(60:\radius) node[above right]{60};
	\draw[#1, \angleColor{90}!80!black!70] (origin) -- +(90:\radius) node[above]{90};
	\draw[#1, \angleColor{120}!80!black!70] (origin) -- +(120:\radius) node[above left]{120};
	\draw[#1, \angleColor{150}!80!black!70] (origin) -- +(150:\radius) node[above left]{150};
	\draw[#1, \angleColor{180}!80!black!70] (origin) -- +(180:\radius) node[left]{180};
	\draw[#1, \angleColor{210}!80!black!70] (origin) -- +(210:\radius) node[left]{210};
	\draw[#1, \angleColor{240}!80!black!70] (origin) -- +(240:\radius) node[below left]{240};
	\draw[#1, \angleColor{270}!80!black!70] (origin) -- +(270:\radius) node[below]{270};
	\draw[#1, \angleColor{300}!80!black!70] (origin) -- +(300:\radius) node[below right]{300};
	\draw[#1, \angleColor{330}!80!black!70] (origin) -- +(330:\radius) node[below right]{330};
}


%\newcommand\polarPlotDrawSkewedRadial[5][lightgray, dashed]{	%  #1:options, #2: angle, #3: origo, #4: x-radius, #5:y-radius
%	\draw[
%		rotate around={#2:#3}, %				rotate 		,#1,
%		#1,
%		] #3 -- #4 node[right]{#2};
%}
%
%\newcommand\polarPlotDrawSkewedRadials[3][lightgray, dashed]{	
%%	\pgfplotsinvokeforeach {0, 60, ..., 330} {
%%		\draw[#1] #2 -- 	+(##1:\radius) node[right]{0};
%}

\def\targetTickDistance{851}


\newcommand\expPolarAxis[2][]{
	
	\def\expThreeRadius{5000}
	\def\expThreeScale{1}
	
	\def\targetTickDistance{851}
	\def\radius{\targetTickDistance * 5 * \strangeScale}
	
	
	%	\def\noYTickLabels{{,,}}
	%	\def\yLabelWhite{\textcolor{white}{}}
	%	\pgfplotsset{every tick label/.append style={font=\scriptsize}}
	%	\pgfplotsset{every tick label/.append style={font=\scriptsize}}		
	\makeTicksFontScriptSize
	
	\begin{axis}[
		%		at={(origin)},
		axis lines=left,
		xtick=\empty, ytick=\empty,
		extra x ticks={-4258, 0, 4258},
		extra x tick labels={ -0.5m, 0, 0.5m},
		extra y ticks={-4258, 0, 4258},
		extra y tick labels={-0.5m, 0, 0.5m},
%		at={(current axis.center)}, %%%%%%%%%%%%%%%%%% FIX APPLIED
%		at={(origin)},
		%		at={(origin)},
		anchor=origin, %center,
		%		anchor=south west,
		xmax=\expThreeRadius,
		ymax=\expThreeRadius,
		xmin=-\expThreeRadius,
		ymin=-\expThreeRadius,
		scale=\expThreeScale,
		%	 		enlarge x limits=false,
		%	 		enlarge y limits=false,	
		%	x=0.5cm, y=0.5cm, 
		%			ymin=-50, ymax=50,		
		%			xmin=-50,xmax=50,
		xmajorgrids=true, 
		ymajorgrids=true,
		enlargelimits=false,
		%		axis equal image, 
		%	ymin=-1, ymax=1,		
		%	%	domain=-50:50,
		%	xmin=-1,xmax=1,
		%	zmin=-1,zmax=1,
		%		nodes near coords,
		%	xlabel={x},
		%	ylabel={z},
		%	xtick distance={10},
		legend style={draw=none},
		axis equal,
		%	legend entries={
		%		 xtick distance={60},
		,#1		 
		]	
		
		\coordinate (origin) at (axis cs:0,0);
		
		
		%			\draw (axis cs:0,0) -- node[left]{Text} (axis cs:200,600);
		%		\draw[->, pink] (axis cs:0,0) -- node[left]{Text} (50:50);
		%			\draw[->, violet] (origin) -- +(75:100);
		
		
		% radials
		
		%			\pgfplotsinvokeforeach {30, 60, ..., 360} {
		%				\draw[lightgray, dashed] (origin) -- +(##1:\radius);
		%			}
		
			
			
		\draw [\targetColor{50},densely dotted] (origin) circle (5 * \targetTickDistance * \strangeScale);
			
		#2		
	\end{axis}	
}

\newcommand\drawCircles[1][]{
	\pgfplotsinvokeforeach {1, ..., 4} {
		\draw [\targetColor{##10}, densely dotted,#1] (origin) circle (##1 * \targetTickDistance * \strangeScale);
	}	
}


\newcommand\releaseSpacePlot[3][]{
	\addplot3[
	%		y={0,<unit length>*cos(10),0},%(<unit length>*cos(60),<unit length>*sin(60))}
	%		rotate around={10:(current axis.origin)},
	%	surf, 
	%	mesh/rows=10, 
	%	shader=interp
	only marks,
	opacity=0.75,
	mark size=1.2,
	%		discard if not={sessionName}{rose}
	%		x filter/.expression={5}
	,#1
	]%surf, mesh/rows=10] 	
	table[
	%			x=trialFinalOffscreenX,
	%			y=trialFinalOffscreenZ,
	%			z=trialFinalOffscreenY,
	x=trialFinalOffscreenLocationWRTDeviceX,
	y=trialFinalOffscreenLocationWRTDeviceY,%trialFinalOffscreenZ,
	z=trialFinalOffscreenLocationWRTDeviceZ,%trialFinalOffscreenY,
	meta=sessionName,#3
	] {#2};% {\expOneCsvFilepath};
}


\newcommand\plotPcaRings{			
	\coordinate (origin) at (axis cs:0,0);			    
	
	%			\begin{scope}[yscale=-1] %, y dir=reverse
	
	%			\def\PPM{85.1692913385827} % i,e. 851
	\def\PPM{10} % i,e. 851
	\foreachTabbedCsvRowInPlot{Pca}{\xc=x, \yc=y,\xr=xr,\yr=yr,\circleAngle=deg,\target=target}{	
		\coordinate (circlePos) at ($(origin) + (\xc/\PPM, \yc/\PPM)$);
		%			\node[mark size=0.02cm,color=\targetColor\target] at (circlePos) {\pgfuseplotmark{triangle*}};
%		\coordinate (circlePos) at ($(origin) + (\xc/\PPM, \yc/\PPM)$);
		\draw (circlePos) node[
			cross,
			rotate around={-1*\circleAngle-45:(circlePos)},
%			rotate around={-1*\circleAngle-45:(circlePos)},
			color=\targetColor\target
			] {};		
		\draw[color=\targetColor\target,very thick] ($(origin) + (\xc/\PPM, \yc/\PPM)$) ellipse [
		x radius=\xr/\PPM,
		y radius=\yr/\PPM,
		rotate around={-1*\circleAngle:(circlePos)} %				rotate around={deg(148):(50,0)}
		];
	}
}

\newcommand\fullPageReleaseSpacePlots[1]{
	\def\leftMarginChange{-2cm}
	\def\myscale{1}
	\vspace*{\leftMarginChange}
	\twoFigure{ 
		\hspace*{\leftMarginChange}
		\begin{tikzpicture}[scale=\myscale,
		%			x label style={at={(axis cs:0.5,-0.1,1)}}
		]%[rotate=90]
		\releaseSpaceAxis[view={10}{17}]{
			%				\begin{scope}[yscale=-1] 
			#1
			%				\end{scope}
		}
		\end{tikzpicture}	
	}{
		\begin{tikzpicture}[scale=\myscale]%[rotate=90]
		\releaseSpaceAxis[view={0}{0}]{ 
			#1
		}
		\end{tikzpicture}		
	}
	\twoFigure{ 	
		\hspace*{\leftMarginChange}
		%		\vspace*{-2cm}
		\begin{tikzpicture}[scale=\myscale]%[rotate=90]
		\releaseSpaceAxis[view={80}{17}]{
			#1
			%				\releaseSpacePlot[\participantColor]{\releaseSpaceData}
		}
		\end{tikzpicture}	
	}{		
		\begin{tikzpicture}[scale=\myscale]%[rotate=90]
		\releaseSpaceAxis[view={90}{0}]{
			#1
		}
		\end{tikzpicture}	
	}	
	\twoFigure{ 	
		\hspace*{\leftMarginChange}
		\begin{tikzpicture}[scale=\myscale]%[rotate=90]
		\releaseSpaceAxis[view={35}{35}]{
			#1
		}
		\end{tikzpicture}	
	}{		
		\begin{tikzpicture}[scale=\myscale]%[rotate=90]
		\releaseSpaceAxis[view={0}{90}]{
			#1					
		}			
		\end{tikzpicture}	
	}	
}

%\NewEnviron{fullPageReleaseSpacePlotsEnv}{
%	\def\leftMarginChange{-2cm}
%	\def\myscale{1.1}
%	\vspace*{\leftMarginChange}
%	\twoFigure{ 
%		\hspace*{\leftMarginChange}
%		\begin{tikzpicture}[scale=\myscale,
%		%			x label style={at={(axis cs:0.5,-0.1,1)}}
%		]%[rotate=90]
%		\releaseSpaceAxis[view={10}{17}]{
%			%				\begin{scope}[yscale=-1] 
%			\BODY
%			%				\end{scope}
%		}
%		\end{tikzpicture}	
%	}{
%	\begin{tikzpicture}[scale=\myscale]%[rotate=90]
%	\releaseSpaceAxis[view={0}{0}]{ 
%		\BODY
%	}
%	\end{tikzpicture}		
%}
%\twoFigure{ 	
%	\hspace*{\leftMarginChange}
%	%		\vspace*{-2cm}
%	\begin{tikzpicture}[scale=\myscale]%[rotate=90]
%	\releaseSpaceAxis[view={80}{17}]{
%		\BODY
%		%				\releaseSpacePlot[\participantColor]{\releaseSpaceData}
%	}
%	\end{tikzpicture}	
%}{		
%\begin{tikzpicture}[scale=\myscale]%[rotate=90]
%\releaseSpaceAxis[view={90}{0}]{
%	\BODY
%}
%\end{tikzpicture}	
%}	
%\twoFigure{ 	
%	\hspace*{\leftMarginChange}
%	\begin{tikzpicture}[scale=\myscale]%[rotate=90]
%	\releaseSpaceAxis[view={35}{35}]{
%	\BODY
%	}
%	\end{tikzpicture}	
%}{		
%\begin{tikzpicture}[scale=\myscale]%[rotate=90]
%\releaseSpaceAxis[view={0}{90}]{
%	\BODY					
%}			
%\end{tikzpicture}	
%}	
%}


%\newcommand\releaseSpacePlotParticipantTargets[2]{
%
%}

\newcommand\releaseSpacePlotAllParticipants[2]{
	\releaseSpacePlot[\leahColor,#1]{\dataTwoLeah}{#2}
	\releaseSpacePlot[\evanColor,#1]{\dataTwoEvan}{#2}
	\releaseSpacePlot[\roseColor,#1]{\dataTwoRose}{#2}
	\releaseSpacePlot[\lilyColor,#1]{\dataTwoLily}{#2}
	\releaseSpacePlot[\soniaColor,#1]{\dataTwoSonia}{#2}
	\releaseSpacePlot[\mikeColor,#1]{\dataTwoMike}{#2}
}




\newcommand\drawInputPlane{
	\addplot3[name path=toppath,fill=blue, opacity=0.1,samples=2, domain=-.5:.5] (x, 0.5, 0);
	\addplot3[name path=botpath,fill=blue, opacity=0.1,samples=2,domain=-.5:.5]  (x, -.5, 0);
	\addplot [gray,opacity=0.02] fill between[of=toppath and botpath];
	\addplot3[fill=blue, opacity=0.1,samples=2, domain=-.5:.5] (0.5, y, 0);
	\addplot3[fill=blue, opacity=0.1,samples=2,domain=-.5:.5]  (-.5, y, 0);
}

\newcommand\drawInputSphere{
	%SPHERE
	\addplot3[surf,
	% opacity = 0.1,
	draw opacity = 0.1,
	fill opacity = 0,
	shader=flat,
	mesh/interior colormap=
	{blueblack}{color=(black) color=(gray)},
	mesh/interior colormap thresh=1,
	% shader=interp,
	samples = 30,
	variable = \u,
	variable y = \v,
	domain = 0:360,
	%        y domain = 0:360,
	]
	({0.3*cos(u)*sin(v)}, {0.3*sin(u)*sin(v)}, {0.3*cos(v)});	
}

\newcommand\releaseSpaceAxis[2][]{
	\makeTicksFontScriptSize
	\begin{axis}[
		enlargelimits=false,
		%	x dir=reverse, 
		xmajorgrids=true, ymajorgrids=true, 
		%	ymin=-1, ymax=1,		
		%	%	domain=-50:50,
		%	xmin=-1,xmax=1,
		%	zmin=-1,zmax=1,
		%		nodes near coords,
		zlabel style={rotate=-90},
		xlabel={x},
		ylabel={y},
		zlabel={z},
		%	xtick distance={10},
		legend style={draw=none},
		%	legend entries={
		%		baseline,
		%		plane-normal,
		%		plane-directional,
		%		spherical},	
		%	legend style={at={(axis cs:-15,15)},anchor=south west},
		%	view={290}{80},
		axis equal
		,#1]
		%	xmax=\domainMax,xmin=-\domainMax,	ymax=\domainMax,ymin=-\domainMax,	
		
		%				\begin{scope}[]%[y dir=reverse]
		#2		
		%				\end{scope}
		
		%		#2		
		
		%	\drawInputSphere\
		
		%		\draw (axis cs:0,0,0) circle (3cm);
		%	\addplot3[name path=toppath,fill=blue, opacity=0.1, fill opacity=0.4,samples=10,domain=0:1] (x, y,0);
		
		%\addplot3 [surf,blue, opacity=0.01] {0};
		
		%\addplot3[contour gnuplot={labels=false,levels={0},draw color=blue},domain=-4:4]
		%{exp(-x^2 -0.3*y^2 -0.2*x*y)*5- 0.1};
		
		\def\circleColor{\myBlue!40}
		
		% helper circles
		\addplot3[\circleColor,samples y=1,,samples=360,domain=1:360] ({0.5*cos(x)},{0.5*sin(x)},0);
		\addplot3[dashed, \circleColor,samples y=1,,samples=360,domain=1:360] (0, {0.5*cos(x)},{0.5*sin(x)});
		\addplot3[dashed, \circleColor,samples y=1,,samples=360,domain=1:360] ({0.5*cos(x)},0,{0.5*sin(x)});
		
		%\addplot[
		%blue,
		%domain=1:pi,
		%samples=17,
		%] 
		
		%	\drawInputPlane
		
		\def\xmin{\pgfkeysvalueof{/pgfplots/xmin}}
		\def\xmax{\pgfkeysvalueof{/pgfplots/xmax}}
		\def\ymin{\pgfkeysvalueof{/pgfplots/ymin}}
		\def\ymax{\pgfkeysvalueof{/pgfplots/ymax}}
		\def\zmin{\pgfkeysvalueof{/pgfplots/zmin}}
		\def\zmax{\pgfkeysvalueof{/pgfplots/zmax}}
		
		%	\drawAxes				
		\draw[thick,gray!30,dotted] (axis cs:\xmin,0,\zmin) -- (axis cs:\xmax,0,\zmin);	
		\draw[thick,gray!30,dotted] (axis cs:0,\ymin,\zmin) -- (axis cs:0,\ymax,\zmin);
		
		\def\linecolor{\myGreen!30}
		
		%	from bottom to device
		\draw[thick,\linecolor, dashed] (axis cs:0,0,\zmin) -- (axis cs:0,0,0);	
		
		%	from device center to axes
		\draw[thick,\linecolor,dotted] (axis cs:\xmin,0,0) -- (axis cs:\xmax,0,0);	
		\draw[thick,\linecolor,dotted] (axis cs:0,\ymin,0) -- (axis cs:0,\ymax,0);
		
		%	vertical lines
		\draw[thick,\linecolor,dotted] (axis cs:0,\ymax,0) -- (axis cs:0,\ymax,\zmin);
		\draw[thick,\linecolor,dotted] (axis cs:0,\ymin,0) -- (axis cs:0,\ymin,\zmin);
		\draw[thick,\linecolor,dotted] (axis cs:\xmin,0,0) -- (axis cs:\xmin,0,\zmin);	
		\draw[thick,\linecolor,dotted] (axis cs:\xmax,0,0) -- (axis cs:\xmax,0,\zmin);	
		
		
		%\addplot3[area plot]  \TL -- \TR -- -- \BR -- \BL -- cycle;
		%
		%-- (axis cs:\pgfkeysvalueof{/pgfplots/xmax},#1,\pgfkeysvalueof{/pgfplots/zmin})
		%-- (axis cs:\pgfkeysvalueof{/pgfplots/xmin},#1,\pgfkeysvalueof{/pgfplots/zmin})
		%-- cycle;			
		
		%% SAN JOSE SPACE:	
		%	\def\TL{(0.223457493299726,-0.16266075098759,1.68620133133376)}	
		%	\def\TR{(0.228854267516688,-0.167606074963504,1.77226374528067)}
		%	\def\BR{(0.184601753473317,-0.196480492341992,1.77337953408869)}
		%	\def\BL{(0.179204979256355,-0.191535168366078,1.68731712014177)}
		% yz swap:
		%	\def\TL{(0.223457493299726, 1.68620133133376, -0.16266075098759)}	
		%	\def\TR{(0.228854267516688, 1.77226374528067, -0.167606074963504)}
		%	\def\BR{(0.184601753473317, 1.77337953408869, -0.196480492341992)}
		%	\def\BL{(0.179204979256355, 1.68731712014177, -0.191535168366078 )}
		
		
		%% SAN JOSE SPACE WRT DEVICE AXIS: 
		Offscreen rect corners WRT device: (TL:), (TR:), (BR:), (BL:)
		
		\def\TL{(-0.043186573421452,-0.0264256597021248,0)}	
		\def\TR{(0.043186573421452,-0.0264256597021248,0)}
		\def\BR{(0.043186573421452,0.0264256597021248,0)}
		\def\BL{(-0.043186573421452,0.0264256597021248,0)}
		
		
		%	\def\TL{(0.0194278699132049,0.0169098706772011,-0.0435891013774647)}	
		%	\def\TR{(0.0248246441301665,0.0119645467012872,0.0424733125694508)}
		%	\def\BR{(-0.0194278699132049,-0.0169098706772011,0.0435891013774647)}
		%	\def\BL{(-0.0248246441301665,-0.0119645467012872,-0.0424733125694508)}
		%SWAP	
		%	\def\TL{(0.0194278699132049, -0.0435891013774647, 0.0169098706772011)}	
		%	\def\TR{(0.0248246441301665, 0.0424733125694508, 0.0119645467012872)}
		%	\def\BR{(-0.0194278699132049,0.0435891013774647, -0.0169098706772011)}
		%	\def\BL{(-0.0248246441301665,-0.0424733125694508, -0.0119645467012872)}
		
		\addplot3[surf, \myRed] coordinates { \TL \TR \BR \BL \TL };
		
		
	\end{axis}			
}

\newcommand\plotReleasePositionsColorByTarget[1][]{
	\addplot[
	shorten >=.05cm,
	shorten <=.05cm,
	scatter,
	only marks,
	opacity=0.75,
	mark size=1.2,
	point meta=explicit symbolic,
	%	scatter/classes=\scatterClasses,
	%			red,
	%			marks only,
	scatter/classes={
		10=\targetColor{10},
		20=\targetColor{20},
		30=\targetColor{30},
		40=\targetColor{40},
		50=\targetColor{50},
		-10=\targetColor{-10},
		-20=\targetColor{-20},
		-30=\targetColor{-30},
		-40=\targetColor{-40},
		-50=\targetColor{-50}
		%			lily={mark=*,\lilyColor,line width=0mm},
		%			Lily={mark=*,\lilyColor,line width=0mm},
		%			leah={mark=*,\leahColor,line width=0mm},
		%			evan={mark=*,\evanColor,line width=0mm},		
		%			evann={mark=*,\evanColor,line width=0mm},		
		%			rose={mark=*,\roseColor,line width=0mm},		
		%			rosee={mark=*,\roseColor,line width=0mm},		
		%					sonia={mark=*,\soniaColor,line width=0mm},
		%					soniaa={mark=*,\soniaColor,line width=0mm},			
		%			unnamed={mark=*,\unnamedColor,line width=0mm},		
		%			mean={mark=none,\meanColor,line width=0mm},
		%			%			mean={mark=none,\meanColor,line width=0mm},
		%			std={mark=none,\stdColor,line width=0mm}
	}
	%	scatter/classes={
	%		mean={mark=*,red,line width=0.5mm},
	%		target={mark=*,blue!80!black,line width=0mm,mark size=1}%%%%%%%%%%%%%%%%%%%%%%%%%%%%%%%%		
	%	}						
	]%surf, mesh/rows=10] 	
	plot[->,error bars/.cd, y dir = both, y explicit]
	table[
	x=trialReleasePositionX,
	y=trialReleasePositionY,
	meta=trialTargetValue,
	%		filter multiple=sessionName equals mean or target,
	%			filter in={trialUIAngleDegrees}{30},
	filter out={trialIsStatistic}{1},#1,
	] 
	{data/exp1/TrialSesssions-(E2-\expParticipants).csv};	
}





%\newcommand{\drawDeviceRect}{
%	\pgfplotsextra{
%		\DTLforeach*{Device}{\onscreenPixelHeight=onscreenPixelHeight, \onscreenPixelWidth=onscreenPixelWidth}{
%			
%		}
%	}
%}


% DATABASES FROM CSV FILES

\def\pcaFilepath{data/exp1/PCA.csv}

\loadTabbedCsv{Pca}\pcaFilepath
\loadTabbedCsv{PcaNoSkew}{data/exp1/PCA-noskew.csv}
\loadTabbedCsv{PcaAggregated}{data/exp1/PCA-aggregated.csv}

%\loadTabbedCsv{Skoooood}\expThreeCsvFilepath


%%% DEVICE PARAMS
\def\deviceFilepath{data/exp1/device.csv}	
\loadTabbedCsv{Device}\deviceFilepath	
%%%




\loadTabbedCsv{PcaTarget50}{data/exp1/PCA-targetValue(50).csv}


\def\PPM{10} % i,e. 851

\newcommand\drawOnscreenDims[1][fill=red!40, draw=red, opacity=0.2]{
	\coordinate (origin) at (axis cs:0,0);			    
	\pgfplotsextra{\DTLforeach*{Device}{\onscreenPixelHeight=onscreenPixelHeight, \onscreenPixelWidth=onscreenPixelWidth}{
			%	\edef\onscreenPixelHeight{\onscreenPixelHeight}
			%	\edef\onscreenPixelWidth{\onscreenPixelWidth}
			\filldraw[#1] ($(origin) - (.5*\onscreenPixelWidth/\PPM,.5*\onscreenPixelHeight/\PPM)$) rectangle ($(origin) +  (.5*\onscreenPixelWidth/\PPM,.5*\onscreenPixelHeight/\PPM)$);	
		}}
	}
	
	
	% E2 July 10: [2016-07-10 20:10:20 INFORMATIONAL] TestPage: Offscreen rect: (0.223457493299726,-0.16266075098759,1.68620133133376) -- (0.228854267516688,-0.167606074963504,1.77226374528067) -- (0.179204979256355,-0.191535168366078,1.68731712014177) -- (0.184601753473317,-0.196480492341992,1.77337953408869)
	
	
	
	
	%\readCsvFiltered{\expTwoCsvFilepath}{
	%%	tablefilter in={sessionName}{evan},
	%%	columns={sessionName, sessionType, trialUIAngle, trialIsStatistic, trialFinalOffscreenX, trialFinalOffscreenY,  trialFinalOffscreenZ},
	%%	tablefilter in={sessionType}{E2},
	%%	tablefilter in={trialUIAngle}{1.0471975511966},
	%%	tablefilter in={sessionName}{Lily},
	%%	tablefilter in={sessionName}{Lily},
	%	tablefilter in={sessionName}{evan},
	%	tablefilter out={trialIsStatistic}{1},
	%%	tablefilter in={trialDurationSecondsIsOutlier}{0}
	%	}\expTwoData
	%%\pgfplotstableread{test.dat}{\expTwoData}
	
	
	
	
	%\pgfplotstablesort[sort key={trialFinalOffscreenY}]{\sorted}{\total} %get the data and sort by column 'T'
	
	%\pgfplotstablesort[sort key={trialFinalOffscreenY}]{\sorted}{\expTwoData} %get the data and sort by column 'T'
	
	
	
	%\def\participantName{Lily}
	%\def\participantName{rose}
	%\def\participantName{evan}
	\def\participantName{rose}
	\def\participantColor{\leahColor!50}
	\pgfplotstableread{data/exp1/TrialSesssions-(E2-\participantName).csv}{\releaseSpaceData}
	
	
	\pgfplotstableread{data/exp1/TrialSesssions-(E2-evan).csv}{\dataTwoEvan}
	\pgfplotstableread{data/exp1/TrialSesssions-(E2-rose).csv}{\dataTwoRose}
	\pgfplotstableread{data/exp1/TrialSesssions-(E2-leah).csv}{\dataTwoLeah	}
	\pgfplotstableread{data/exp1/TrialSesssions-(E2-Lily).csv}{\dataTwoLily}
	\pgfplotstableread{data/exp1/TrialSesssions-(E2-soniaa).csv}{\dataTwoSonia}
	\pgfplotstableread{data/exp1/TrialSesssions-(E2-mikee).csv}{\dataTwoMike}
	
	
	\newcommand\expPolarPlotAngles{
		\expPolarPlotAngle{0}
		\expPolarPlotAngle{30}
		\expPolarPlotAngle{60}
		\expPolarPlotAngle{90}
		\expPolarPlotAngle{120}
		\expPolarPlotAngle{150}
	}
	
%	\newcommand\expPolarPlotAnglesDimmed[1]{
%		\expPolarPlotAngle[#1]{0}
%		\expPolarPlotAngle[#1]{30}
%		\expPolarPlotAngle[#1]{60}
%		\expPolarPlotAngle[#1]{90}
%		\expPolarPlotAngle[#1]{120}
%		\expPolarPlotAngle[#1]{150}
%	}

	
%\usepackage{floatrow}			

\usepackage{pdfpages}

\newcommand\expThreePlotDiagonals[1][lightgray, dashed]{
		\draw[#1] ($(origin) + (\angleAx/\PPM,\angleAy/\PPM)$) -- ($(origin) + (\angleGx/\PPM,\angleGy/\PPM)$);
		\draw[#1] ($(origin) + (\angleBx/\PPM,\angleBy/\PPM)$) -- ($(origin) + (\angleHx/\PPM,\angleHy/\PPM)$);
		\draw[#1] ($(origin) + (\angleCx/\PPM,\angleCy/\PPM)$) -- ($(origin) + (\angleIx/\PPM,\angleIy/\PPM)$);
		\draw[#1] ($(origin) + (\angleDx/\PPM,\angleDy/\PPM)$) -- ($(origin) + (\angleJx/\PPM,\angleJy/\PPM)$);
		\draw[#1] ($(origin) + (\angleEx/\PPM,\angleEy/\PPM)$) -- ($(origin) + (\angleKx/\PPM,\angleKy/\PPM)$);
		\draw[#1] ($(origin) + (\angleFx/\PPM,\angleFy/\PPM)$) -- ($(origin) + (\angleLx/\PPM,\angleLy/\PPM)$);
	}

\newcommand\expThreeForeachPcaRow[3][1]{ %\DTLiseq{\target}{50}
			\pgfplotsextra{\DTLforeach*[#1]{#2}{\xc=x, \yc=y,\xr=xr,\yr=yr,\circleAngle=deg,\target=target,\circleRadius=circleRadius,\circleRadiusM=circleRadiusM,
					\angleAx=angle0x,
					\angleAy=angle0y,
					\angleBx=angle30x,
					\angleBy=angle30y,
					\angleCx=angle60x,
					\angleCy=angle60y,
					\angleDx=angle90x,
					\angleDy=angle90y,
					\angleEx=angle120x,
					\angleEy=angle120y,
					\angleFx=angle150x,
					\angleFy=angle150y,
					\angleGx=angle180x,
					\angleGy=angle180y,
					\angleHx=angle210x,
					\angleHy=angle210y,
					\angleIx=angle240x,
					\angleIy=angle240y,
					\angleJx=angle270x,
					\angleJy=angle270y,
					\angleKx=angle300x,
					\angleKy=angle300y,
					\angleLx=angle330x,
					\angleLy=angle330y}
	{
		\coordinate (circlePos) at ($(origin) + (\xc/\PPM, \yc/\PPM)$);

		#3
	}}	
}


\newcommand\polarPlotDrawRadialsReversed{
			\begin{scope}[yscale=-1] 
				\polarPlotDrawRadials[lightgray, dashed, shorten >=.03cm, shorten <=.4cm, opacity=0.3] 
			\end{scope}			
}

\newcommand\expThreePlotComfortZone[1]{
	%% ELLIPSES
	\expThreeForeachPcaRow[\DTLiseq{\target}{50}]{#1}{				
		\draw[\targetColor{50}, thick, opacity=0.8, fill=\targetColor{50}!30!black, fill opacity=0.02, dashed] ($(origin) + (\xc/\PPM, \yc/\PPM)$) ellipse [
		x radius=\xr/\PPM,
		y radius=\yr/\PPM,
		rotate around={-1*\circleAngle:(circlePos)} 
		];
	}
}


\newcommand\expThreePlot[3]{
	%	\begin{figure}	
	%		\centering
		%		\hspace*{-1cm}			
		%		\twoFigure{		
		%			%		\hspace*{-0.22cm}
		\begin{tikzpicture}[scale=1.2,>=stealth]
		%		\hspace*{-0.22cm}
		\expPolarAxis[		
		extra y tick labels={ 0.5m, 0, -0.5m},					
		y dir=reverse,  %%% REVERSE BECAUSE TRACKER VIEWS FROM OPPOSITE SIDE!		
		]{	
			
%			\polarPlotDrawRadialsReversed			
			%		\begin{scope}[yscale=-1] 
			%		\polarPlotDrawRadials 
			%		\end{scope}			
			%		
			
			#3
											
			\coordinate (origin) at (axis cs:0,0);			    

%			\expThreePlotComfortZone{#2}

			\expThreeForeachPcaRow[\DTLiseq{\target}{50}]{#2}{				
					%% RADIALS
					%%				, path fading=west, fading angle=30
					%				\newcommand\drawSkewedRadials{		
					
					% angle numbers		
					\node[draw=none,lightgray, right] at ($(origin) + (\angleAx/\PPM,\angleAy/\PPM)$) {\scriptsize 0};
					\node[draw=none,lightgray, above right] at ($(origin) + (\angleBx/\PPM,\angleBy/\PPM)$) {\scriptsize 30};
					\node[draw=none,lightgray, above right] at ($(origin) + (\angleCx/\PPM,\angleCy/\PPM - 3)$) {\scriptsize 60};
					\node[draw=none,lightgray, above] at ($(origin) + (\angleDx/\PPM,\angleDy/\PPM - 12)$) {\scriptsize 90};
					\node[draw=none,lightgray, above left] at ($(origin) + (\angleFx/\PPM,\angleFy/\PPM)$) {\scriptsize 150};
					\node[draw=none,lightgray, above left] at ($(origin) + (\angleEx/\PPM,\angleEy/\PPM)$) {\scriptsize 120};
					\node[draw=none,lightgray, left] at ($(origin) + (\angleGx/\PPM,\angleGy/\PPM)$) {\scriptsize 180};
					\node[draw=none,lightgray, below left] at ($(origin) + (\angleHx/\PPM,\angleHy/\PPM)$) {\scriptsize 210};
					\node[draw=none,lightgray, left] at ($(origin) + (\angleIx/\PPM,\angleIy/\PPM)$) {\scriptsize 240};
%					\node[draw=none,lightgray, below] at ($(origin) + (\angleJx/\PPM,\angleJy/\PPM)$) {\scriptsize 270};
					\node[draw=none,lightgray, below right] at ($(origin) + (\angleKx/\PPM,\angleKy/\PPM)$) {\scriptsize 300};
					\node[draw=none,lightgray, below right] at ($(origin) + (\angleLx/\PPM,\angleLy/\PPM)$) {\scriptsize 330};
					
%					\expThreePlotDiagonals[\targetColor{\ta}]
				}
				%\shade[inner color=transparent,outer color=red!40] (0,0) rectangle (4,4);
				
				
%				\drawOnscreenDims
				
				%			\expPolarPlotAngles
				%			\plotReleasePositionsColorByTarget			
			}	
			\end{tikzpicture}
%			\caption{Determining the radius, cutoff target=40 TODO!}
%			\label{fig:expThreeSkew}				
%		\end{figure}
	}	

\newcommand\expThreePlotRadials[2][black!50, dashed,mark size=0.07]{
%	\expThreePlot{#1}{
			% RADIALS			
			\addplot[
			%			red,
			scatter,
			only marks,
			mark size=0.1,
			%		red,
			point meta=explicit symbolic,
			,
			scatter/classes={
				%			ellipse=transparent, %{mark=*,gray,line width=1mm}							
%				0={\targetColor{50}!50, dashed} %{mark=*,gray,line width=1mm}					
				1={#1} %{mark=*,gray,line width=1mm}					
			}					
			] 	
			table[
			x=x,
			y=y,	
			meta=isRadial,
%			filter in={type}{radial},		
			filter in={isRadial}{1},								
%			filter in={target}{50},					
			%		filter in={target}{50}					
			] {#2};	
%	}
}

\newcommand\comfotZoneBisect[5][gray!50, dashed]{
	%% ELLIPSES 			
	\addplot[
	%			red,
	scatter,
	%			only marks,
	opacity=0.75,
	mark size=0.3,
	point meta=explicit symbolic,
	%				scatter/classes={
	%					0=\targetColor{10},
	%					30=\targetColor{20},
	%					60=\targetColor{30},
	%					90=\targetColor{40},
	%					120=\targetColor{50} %				50={mark=*,\targetColor{50},line width=1mm, draw=\targetColor{50}}							
	%				}	
	,#1,				
	] 	
	table[
	x=x,
	y=y,	
	meta=deg
	,filter in={target}{#5}					
	,filter multiple={deg equals #2 or #3}					
	%			filter in={target}{50}					
	] {#4};	
}

\newcommand\expThreePlotEllipse[2][]{			
	%			%% ELLIPSES 			
	\addplot[
	scatter,
	only marks,
	opacity=0.75,
	mark size=0.3,
	point meta=explicit symbolic,
	scatter/classes={
		10=\targetColor{10},
		20=\targetColor{20},
		30=\targetColor{30},
		40=\targetColor{40},
		50=\targetColor{50} %				50={mark=*,\targetColor{50},line width=1mm, draw=\targetColor{50}}							
	}					
	] 	
	table[
	x=x,
	y=y,	
	meta=target
	,filter in={isEllipse}{1}
	,#1,					
	%			filter in={target}{50}					
	] {#2};	
}

\usepackage{pgfplotstable}

\makeatletter
\pgfplotsset{
	/pgfplots/flexible xticklabels from table/.code n args={3}{%
		\pgfplotstableread[#3]{#1}\coordinate@table
		\pgfplotstablegetcolumn{#2}\of{\coordinate@table}\to\pgfplots@xticklabels
		\let\pgfplots@xticklabel=\pgfplots@user@ticklabel@list@x
	}
}
\makeatother

\newcommand\expPolarPlotReleaseDiff[5][]{%%%%%%%%TEST
	%	\pgfmathdivide{#1}{180}
	%	\definecolor{degreeColor#1}{hsb}{\pgfmathresult, 1, 1}


	\addplot[
		%		bar shift=12.5,
		%		red,
		%			only marks,
		%	->,
		rotate around={#2:(axis cs:0,0)},
		%	shorten >=.15cm,
		%	shorten <=.15cm,
		ycomb,
		nodes near coords,
		nodes near coords={
			\pgfmathparse{abs(\mean/8516.92913385827)} 
			\pgfmathprintnumberto[precision=2]{\pgfmathresult}{\roundedmean} 
			\pgfmathparse{\std/8516.92913385827)} 
			\pgfmathprintnumberto[precision=2]{\pgfmathresult}{\roundedstd} 
			\pgfmathparse{0.3 + abs(\target)/100} 
			\scalebox{\pgfmathresult}{ \roundedmean / \roundedstd  } % \trialUIAngleDegrees }% \roundedmean / \roundedstd }
%			\pgfmathresult
		},
		visualization depends on={value \thisrow{#4} \as \mean},
		visualization depends on={value \thisrow{#4Confidence} \as \std},
		visualization depends on={value \thisrow{trialTargetValue} \as \target},
		visualization depends on={value \thisrow{trialUIAngleDegrees} \as \trialUIAngleDegrees},
%			\ifthenelse{\equal{\thecitation}{This work}}{\pgfmathparse{\thenumber*100}\pgfmathprintnumber{\pgfmathresult}\% [This work]}{\pgfmathparse{\thenumber*100}\pgfmathprintnumber{\pgfmathresult}\% \cite{\thecitation}}
%		},		
%		nodes near coords align={above left},	
		every node near coord/.append style={
%			rotate around={#2-90:(axis cs:0,0)},
			      /pgf/number format/fixed,
			      /pgf/number format/precision=5,
			      rotate=#2-90,%{Mod(#2,180)}, 
			anchor=#5, %center
			inner sep=2pt,
			scale=0.7,
%			scale=9.
			font=\tiny % \scriptsize\selectfont,							
		},
%		xtick=data,
%		xticklabel style={
%				text height=1.5ex
%				rotate=90, 
%				font=\scriptsize\selectfont,							
%%				scale=0.5 
%		},
%		symbolic y coords={trialFromTargetToReleaseVerticalLength},
%		flexible xticklabels from table={\expTwoPolarCsvFilepath}{trialFromTargetToReleaseVerticalLength}{col sep=tab},
		%	x dir=reverse, 
		%	scatter,
		ybar, 
		bar width=2,
		fill=#3,%degreeColor#1!90,%red!30,
%		point meta=explicit symbolic,
		%			red,
		%	scatter/classes={
		%		mean={mark=*,\myRed,line width=0mm},
		%		target={mark=*,\myGreen,line width=0mm}		
		%	}			
%		show sum on top,
		,#1,			
		]%surf, mesh/rows=10] 	
		plot[error bars/.cd, 
			y dir = both, y explicit,  	
			%	x dir = both, x explicit,  	
			%		error bar style={
			%			line width=1.5pt, 
			%			rotate around={30:(axis cs:\x,\y)},
			%			xshift=4.5mm
			%			},
			error mark options = {
				%		rotate around={#1:current origin} 
				%%		line width=1.5pt, 
				mark size = 0pt 
			},
			] %,bar width=3pt,
		table[
			x=trialTargetHorizontalDistanceFromOrigo,
	%		x expr=\coordindex,
			y=#4,
			y error=#4Confidence,
			%	x error=trialFromTargetToReleaseVerticalLengthConfidence,
			meta=sessionName,
			filter in={sessionName}{mean},
			%			filter in={sessionType}{E2},
			filter in={trialUIAngleDegrees}{#2},
			%	filter in={trialTargetValue}{#2}
			%		filter in={trialUIAngle}{1.5707963267949}
			] 
			%			{data/exp1/TrialSesssions-(E2-polar-evan,rose,Lily,Lily).csv} 
			{\expTwoPolarCsvFilepath};	
}




\newcommand\improvement[2]{\round[2]{100*(#1/#2-1)}}

% EXP 3 -----------------------------

%\newcommand\expThreeStatsPlot[2]{
%	\addplot[		
%	%			domain=-50:50,
%	%		error band,
%	%			visualization depends on=\thisrow{alignment} \as \alignment,
%	%			every node near coord/.style={anchor=\alignment} ,
%	%			nodes near coords*={(5,5)},
%	%			nodes near coords*={$(\pgfmathprintnumber[frac]\myvalue)$},
%	%			visualization depends on={\thisrow{targetValue} \as \myvalue}
%	%			clickable coords={\thisrow{label}}, 
%	%		color=red,
%	mark size=0.5mm,
%	enlarge x limits=true,
%	enlarge y limits=true,
%	scatter,
%	color=\meanColor,
%	%			scatter src=explicit symbolic, 
%	point meta=explicit symbolic,
%	%		point meta=explicit symbolic,
%	scatter/classes={
%		lily={mark=*,\lilyColor,line width=0mm},
%		Lily={mark=*,\lilyColor,line width=0mm},
%		leah={mark=*,\leahColor,line width=0mm,xshift=-0.05cm},
%		evan={mark=*,\evanColor,line width=0mm,xshift=0.05cm},		
%		evann={mark=*,\evanColor,line width=0mm,xshift=0.05cm},		
%		rose={mark=*,\roseColor,line width=0mm,xshift=0.1cm},		
%		rosee={mark=*,\roseColor,line width=0mm,xshift=0.1cm},
%		sonia={mark=*,\soniaColor,line width=0mm,xshift=-0.1cm},
%		soniaa={mark=*,\soniaColor,line width=0mm,xshift=-0.1cm},
%		mike={mark=*,\mikeColor,line width=0mm,xshift=-.15cm},
%		mikee={mark=*,\mikeColor,line width=0mm,xshift=-.15cm},
%		unnamed={mark=*,\unnamedColor,line width=0mm,xshift=0.15cm},		
%		mean={mark=none,\meanColor,line width=0mm},
%		%			mean={mark=none,\meanColor,line width=0mm},
%		std={mark=none,\stdColor,line width=0mm}
%	},
%	%			unbounded coords=skip
%	,#1
%	] 	
%	plot[error bars/.cd, y dir = both, y explicit]
%	table[
%	meta=sessionName,
%	%			  col sep = \csvSep,
%	%		x=#2,	%trialTargetValue,
%	%		%			  y=trialTargetValue,
%	%		%			  y=trialReleasePositionTargetDiff
%	%		y=#3,
%	%		y error=#4	
%	%		#2] {\dataSource}; %\expThreeCsvFilepath};
%	#2] {\expThreeCsvFilepath};
%}
%
%






















%\begin{filecontents*}{sinePlaneFitPoints.csv}
%\end{filecontents*}




