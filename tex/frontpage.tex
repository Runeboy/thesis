
\title{
	Off-screen Interactions Beyond the Display \\Boundaries of Portable Devices\\}
%	\numberofauthors{1} 
	\author{
		Author: Rune Munkholm Jensen\\
		Supervisor: \\% Sebastian Boring\\
		Institute: DIKU\\
	}
%\date{}

%\maketitle

\thispagestyle{empty}

\renewenvironment{abstract}
{\small
	\begin{center}
		\bfseries \abstractname\vspace{0cm}\vspace{0pt}
	\end{center}
	\list{}{%
		\setlength{\leftmargin}{4.78cm}% <---------- CHANGE HERE
		\setlength{\rightmargin}{\leftmargin}%
	}%
	\item\relax}
{\endlist}

\hspace{5cm}

\begin{abstract}
%	\lipsum[1]
Much of today's human-computer interaction takes place on portable displays that are based solely on using touch as input. However, since smaller displays are not able to present much information, navigational techniques must be available for the user to efficiently navigate a potentially large information space, which may only be partially in view. But if these techniques rely on touch alone, it creates a similar problem, because the display plane itself provides very limited input space and has an easily obstructed view. What results are redundant inputs for invoking highly similar actions, possibly with a high use of pseudo-input (such as inertia) for maintaining a clear line of  view.  Hence, if a mobile touch-based interface could extend beyond its input boundaries and into an imaginary extension of the display plane, it would likely eliminate much of this input redundancy, leading to fewer touch operations and presumably open up for more efficient navigation of the user interface.

Using three-dimensional space  for input has recently become a  highly active research field, one with great potential in the multitude of cataclysms that are currently unfolding in all areas of computer science. The paper here contributes to the research body with an exploration of one potential usage strategy for exploiting the space surrounding a mobile device,  done so through a solution based on related work and implemented using current technology. To measure up its applicability, test subjects perform a sequence of navigational tasks that incorporates this space and the result is compared to a baseline that does not. What is seen, is that such  baseline is hard to beat in terms of interaction time, but that one particular definition of the off-screen space comes very close and in addition, greatly enhances the overall user experience. Resulting questions are then explored further. The test subjects' perceptions of off-screen space are evaluated through another experiment and the results subsequently used to modify the original solution. A final experiment then shows how this brings about a modest performance improvement in interaction time, when compared to the original solution, as well as potential pathways to take in future work.% Hence, the work here 

% the off-screen space has a certain shape and a subsequent attempt at conforming to this yields an optimization that comes even closer to the baseline. Hence, the work undertaken here provides a proven strategy that enhances the user experience and one that may be optimized further in future studies.  

	
	
\end{abstract}


\newpage

\begin{comment}
\section{Acknowledgements}
A thanks to Sebastian Boring for his insightful comments and sharing of perspective  during the learning process of this work.
\newpage
\end{comment}


\begin{comment}

\section*{Abbreviations}
\begin{itemize}
	\item HCI: Human computer interaction
\end{itemize}

\newpage

\end{comment}

%  summations take first and	bla bla is the 2-norm of a vector and bla bla is the 2-norm of a matrix ....... || is size of vector, || || is 2-norm Euclidean length


%\section*{Table of Contents}
%\renewcommand{\contentsname}{}
\begin{adjustwidth}{3cm}{3cm}
\centering
\tableofcontents
\end{adjustwidth}